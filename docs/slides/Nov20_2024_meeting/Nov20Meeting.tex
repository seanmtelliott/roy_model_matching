%%%%%%%%%%%%%%%%%%%%%%%%%%%%%%%%%%%%%%%%%
% Beamer Presentation
% LaTeX Template
% Version 1.0 (10/11/12)
%
% This template has been downloaded from:
% http://www.LaTeXTemplates.com
%
% License:
% CC BY-NC-SA 3.0 (http://creativecommons.org/licenses/by-nc-sa/3.0/)
%
%%%%%%%%%%%%%%%%%%%%%%%%%%%%%%%%%%%%%%%%%

%----------------------------------------------------------------------------------------
%	PACKAGES AND THEMES
%----------------------------------------------------------------------------------------

\documentclass[aspectratio=169]{beamer}
\beamertemplatenavigationsymbolsempty
\let\oldfootnotesize\footnotesize
\renewcommand*{\footnotesize}{\oldfootnotesize\tiny}

\useinnertheme[shadow=false]{rounded} 
\useoutertheme{infolines} 
%\usecolortheme{beaver}  
\setbeamerfont{block title}{size={}} 
\setbeamercolor{titlelike}{parent=structure,bg=white}
\mode<presentation> {
	
	% The Beamer class comes with a number of default slide themes
	% which change the colors and layouts of slides. Below this is a list
	% of all the themes, uncomment each in turn to see what they look like.
	
	%\usetheme{default}
	%\usetheme{AnnArbor}
	%\usetheme{Antibes}
	%\usetheme{Bergen}
	%\usetheme{Berkeley}
	%\usetheme{Berlin}
	%\usetheme{Boadilla}
	%\usetheme{CambridgeUS}
	%\usetheme{Copenhagen}
	%\usetheme{Darmstadt}
	\usetheme{Dresden}
	%\usetheme{Frankfurt}
	%\usetheme{Goettingen}
	%\usetheme{Hannover}
	%\usetheme{Ilmenau}
	%\usetheme{JuanLesPins}
	%\usetheme{Luebeck}
	%\usetheme{Madrid}
	%\usetheme{Malmoe}
	%\usetheme{Marburg}
	%\usetheme{Montpellier}
	%\usetheme{PaloAlto}
	%\usetheme{Pittsburgh}
	%\usetheme{Rochester}
	%\usetheme{Singapore}
	%\usetheme{Szeged}
	%\usetheme{Warsaw}
	
	% As well as themes, the Beamer class has a number of color themes
	% for any slide theme. Uncomment each of these in turn to see how it
	% changes the colors of your current slide theme.
	
	%\usecolortheme{albatross}
	%\usecolortheme{beaver}
	%\usecolortheme{beetle}
	%\usecolortheme{crane}
	\usecolortheme{dolphin}
	%\usecolortheme{dove}
	%\usecolortheme{fly}
	%\usecolortheme{lily}
	%\usecolortheme{orchid}
	%\usecolortheme{rose}
	%\usecolortheme{seagull}
	%\usecolortheme{seahorse}
	%\usecolortheme{whale}
	%\usecolortheme{wolverine}
	
	%\setbeamertemplate{footline} % To remove the footer line in all slides uncomment this line
	\setbeamertemplate{footline}[page number] % To replace the footer line in all slides with a simple slide count uncomment this line
	
	%\setbeamertemplate{navigation symbols}{} % To remove the navigation symbols from the bottom of all slides uncomment this line
}

%% Packages
\usepackage[round]{natbib} %Allows custom options for bibliography management
\usepackage{cite} %Adds bibliograhy functionality like \citet
\usepackage{amsmath} %Adds additional math-related commands
\usepackage{amsfonts} %Adds fonts-within-math-mode functionality
\usepackage{amsthm} %Enables custom theorem environments
%\usepackage{bm} %Adds bold math fonts
\usepackage{adjustbox} %Need to adjust the size of tikz diagram
\usepackage{tikz} %Used for formatting plots and figures
\usepackage{float} %Used to force the position of tables and figures
\usepackage{graphicx} %For including images
\usepackage[T1]{fontenc} %Allows for encoding of slavic letters
\usepackage{lmodern} %Need to load back the original font for when T1 encoding is active
\usepackage{caption}
%%


%%% User-defined commands

\DeclareMathOperator*{\argmax}{argmax} % The argmax operator
\DeclareMathOperator\supp{supp}
\newtheorem{assumption}{Assumption} % Treating assumptions like theorems.
\newtheorem{proposition}{Proposition} % Treating propositions like theorems.

%%%

\bibliographystyle{apalike}
% make bibliography entries smaller
\renewcommand\bibfont{\scriptsize}
% If you have more than one page of references, you want to tell beamer
% to put the continuation section label from the second slide onwards
\setbeamertemplate{frametitle continuation}[from second]
% Now get rid of all the colours
\setbeamercolor*{bibliography entry title}{fg=black}
\setbeamercolor*{bibliography entry author}{fg=black}
\setbeamercolor*{bibliography entry location}{fg=black}
\setbeamercolor*{bibliography entry note}{fg=black}
% and kill the abominable icon
\setbeamertemplate{bibliography item}{}


\begin{document}
	\section{Introduction}
\begin{frame}{Roy model identification---\citet{heckman1990empirical}}
	\begin{itemize}
	\item Each individual's skill level is given by the ordered pair $(k,s)$ where $k\in\mathbb{R}_+$ and $s\in\mathbb{R}_+$ represent the managerial and worker skill, respectively. 
	\bigskip
	\item Individual's are paid according to their skill level with wage functions $\pi(k) = k$ and $w(s) = s$.
	\bigskip
	\item Workers will choose the occupation that maximizes their wage.
	\bigskip
	\item Assume that $(\ln k, \ln s) \sim N(\mathbf{\mu},\Sigma)$ where $\mu = [\mu_k,\mu_s]'$ and $\Sigma=\begin{bmatrix}
		\sigma_k^2 & \sigma_{ks} \\
		\sigma_{ks} & \sigma_s^2
	\end{bmatrix}$ 
	\bigskip
	\item Denote the joint distribution of skills as $\mathcal{R}\subseteq \mathbb{R}^2_+$---this is the object we wish to identify. 
	\bigskip
	\item The challenge is that we only observe the wage distribution conditional on occupational choice.
	\end{itemize}
\end{frame}

	\begin{frame}{Identification of the log-normal Roy model}
	\begin{theorem}
		Given the model outlined on the previous slide, we have the following results from \citet{heckman1990empirical}:
		\begin{itemize}
			\item HH Thm. 4: If we observe wages for both occupations and occupational choice, both $\mathbf{\mu}$ and $\Sigma$, and their subscripts, are identified.
			\smallskip
			\item  HH Thm. 5: If we only observe the aggregate earnings distribution, we can identify both $\mathbf{\mu}$ and $\Sigma$ but not their subscripts.\footnote{This corresponds to the case where wages in one sector---typically thought of as the household sector---are unobserved.}
			\smallskip
			\item HH Thm. 6: If we observe earnings for one occupation, say $k$, it is possible to identify $\mu_k$, $\sigma_k$, $\frac{\mu_k - \mu_s}{\sigma}$ and $\rho$ where $\sigma = \sqrt{\sigma_k + \sigma_s -2 \sigma_{ks}}$ and $\rho = \frac{\sigma_k - \sigma_{ks}}{\sigma\sqrt{\sigma_k}}$. However, we cannot identify $\mu_s$ and $\sigma$.
		\end{itemize}
	\end{theorem}
\end{frame}

\begin{frame}{Beyond log-normality}
	\begin{itemize}
		\item Relaxing normality implies that sectoral wage data can be rationalized by a model with skills having arbitrary correlation. \citep{heckman1990empirical}
		\bigskip
		\item However, this can be remedied under different approaches:
		\begin{itemize}
			\smallskip
			\item Identification at infinity: \citet{heckman1990empirical} \citet{french2011identification}, \citet{d2013another}
			\smallskip
			\item Adding a non-pecuniary component: \citet{heckman2007econometric} \citet{d2013inference}, \citet{bayer2011nonparametric}, \citet{lee2023nonparametric}.
			\smallskip
			\item Exclusion restrictions: \citet{heckman1990empirical}, \citet{buera2006non}, \citet{french2011identification}, \citet{mourifie2020sharp}
		\end{itemize}
		\bigskip
		\item We can take a different approach: suppose we also observe firms who match workers to maximize output. Does this give us any identification power?
	\end{itemize}
\end{frame}

\section{Normal Roy model with matching}

\begin{frame}{Applying HH to the case with matching}

Introduce a firm with a strictly supermodular production function $F:\mathbb{R}^2_+ \to \mathbb{R}$ which depends on the type of workers $k$ and $s$ they employ.

\vspace{0.3cm}

	Firms choose a worker of each type to maximize profit given by
\begin{equation*}
	\max_{(k,s)\in\supp(\mathcal{R})} F(k,s) - \pi(k) - w(s).
\end{equation*}

The simplest possible case is where
\begin{equation*}
	F(k,s) = k + s + ks
\end{equation*}

as it implies the matching function is known and will be the identity map
\begin{equation*}
	\mu(k) = k.
\end{equation*}

\end{frame}

\begin{frame}{The wage functions}
	 It can be shown that, given knowledge of $F(k,s)$, the wage functions are also known and given by
	\begin{equation*}
		\pi(k) = k + \int_0^k\mu(\tilde{k})d\tilde{k} \text{ and } w(s) = s + \int_0^s\mu^{-1}(\tilde{s})d\tilde{s}.
	\end{equation*}
 	Since the matching function is known this case, we can say that
	\begin{equation*}
		\pi(k) = k + \int_0^k\tilde{k}d\tilde{k} = k +\frac{k^2}{2}
	\end{equation*}
	is the manager's wage and the worker's wage is 
	\begin{equation*}
		w(s) = s + \frac{s^2}{2}
	\end{equation*}
	by symmetry.
\end{frame}


\begin{frame}{Identification in \citet{heckman1990empirical}}
	
Parametric identification comes from the fact that the wages are linear functions of normally distributed random variables. 

\vspace{0.2cm}

That is, occupational choice for individual $i$ is given by
\begin{equation*}
Pr(J_i = k)=	Pr[\pi(k)>w(s)] = Pr(k>s) = \frac{\mu_k - \mu_s}{\sqrt{\sigma_k^2 + \sigma_s^2 - 2\sigma_{ks}}}.
\end{equation*}
Let $Y_i$ denote the wage for individual $i$. 

\vspace{0.5cm}

 We have closed-form expressions for observed conditional moments: $E(Y_i | J_i = k)$, $E(Y_i | J_i = s)$, $Var(Y_i | J_i = k)$, $Var(Y_i | J_i = s)$, $E([Y_i - E(Y_i|J_i = k)]^3|J_i = k)$, and $E([Y_i - E(Y_i|J_i = s)]^3|J_i = s)$ and can uncover the parameters of the distribution exactly.

\end{frame}

\begin{frame}{Back to the matching case}
	We observe the following:
	\smallskip
\begin{itemize}
	\item Occupational choice probabilities: $Pr(J_i = k)=Pr(S_i = s)=1/2$
	\item Moments of the conditional wage distribution: $E[(Y_i - \mu_{Y|K})^n| J_i = k]$ and $E[(Y_i - \mu_{Y|S})^n| J_i = s]$.
	\item Moments of the conditional revenue distribution: $E[(F(k_i,s_j)-\mu_{F|K,S})^n|J_i = k,J_j = s]$.
\end{itemize}

\vspace{0.25cm}

However, even though $k$ and $s$ are normally distributed, the wages are not. $\pi(k)=k+k^2/2$

\vspace{0.5cm}

Therefore, it is worth asking: is the parametric case is interesting? It is a strong assumption which doesn't seem to buy us much in the way of simplicity.
\end{frame}


\section{The $2\times 2$ Model}

\begin{frame}{The $2\times 2$ Roy model without matching}
	\begin{itemize}
		\item Workers are endowed with a skill $k\in\{k_L,k_H\}=K$ and a skill $s\in\{s_L,s_H\}=S$ such that each worker can be characterized by the ordered pair $(k,s)\in K\times S$.
		\bigskip
		\item As before, we assume that workers get paid their skill level.
		\bigskip
		\item I will introduce a few assumptions and establish a non-identifiability result
		\begin{itemize}
			\item Strict monotonicity: Assume wlog $k_H > s_H > k_L > s_L$.
			\smallskip
			\item Occupational choice: Workers choose the occupation $k$ or $s$ which maximizes their wage.
			\smallskip
			\item Full information: We observe occupational choice and wages in both sectors.
		\end{itemize}
		\bigskip
		\item In this setting, the joint distribution of skills is unidentified.
	\end{itemize}
\end{frame}

\begin{frame}{Lack of identification in the basic $2\times 2$ model}
	\begin{itemize}
		\item Let $n(k,s)$ denote the count of types $(k,s)\in K\times S$ in the population---this is the object of interest.
		\smallskip
		\item Using the assumptions on the previous slide, we can establish a few facts:\footnote{Recall the strict monotonicity assumption: $k_H > s_H > k_L > s_L$. We could write this inequality in a different order and establish a similar result.}
		\begin{enumerate}
			\item If a worker chooses $s_H$, they must be $k_L$ $\implies$ we know $n(k_L,s_H)$.
			\item If a worker chooses $k_L$, they must be $s_L$ $\implies$ we know $n(k_L,s_L)$.
			\item No worker chooses $s_L$.
			\item If a worker chooses $k_H$, they could be either $s_H$ or $s_L$.
		\end{enumerate}
		\smallskip
		\item We know that $n(k_H) = n(k_H,s_H) + n(k_H,s_L)$ but we cannot identify these two quantities separately.
	\end{itemize}
\end{frame}


\begin{frame}{Introduction of a firm \& matching}
	\begin{itemize}
		\item I'm going to show that even with matching the non-identification result persists and I'll explain why.
		\bigskip
		\item Suppose we have a firm with a strictly supermodular production function $F:K\times S \to \mathbb{R}_+$ and that $k$ workers are more impactful in production.\footnote{For example, we could consider $F(k,s)=k^2s$. This is the production function in \citet{kremer1996wage}.}
		\bigskip
		\item Firms operate in a perfectly competitive market with free entry implying all output is split between the workers.
		\bigskip
		\item In this setting, the share of the output which goes to each worker---i.e., their wage---also depends on the scarcity of their type in the population.
	\end{itemize}
\end{frame}

\begin{frame}{$2\times 2$ model with matching}
		 I'll introduce a couple of assumptions about the underlying skill distributions---these are fairly natural, I think. This is just to focus the problem a little bit.
 
 		\bigskip
 
		\begin{itemize}
			\item Talent scarcity: Assume that high-skill managers are more scarce than high-skill workers.
			\begin{equation*}
				n(k_H) < n(s_H)
			\end{equation*}
			\item Positive correlation: Assume that if a worker is high/low skill in one occupation they are also more likely to be high/low skill in the other.
			\begin{equation*}
				\sum_{\tau \in \{L,H\}}n(k_\tau,s_\tau) > \sum_{\tau \in \{L,H\}}n(k_\tau,s_{\sim\tau})
			\end{equation*}
		\end{itemize}

\end{frame}



\begin{frame}
	    \begin{columns}
		\begin{column}{0.4\textwidth}
			\begin{table}
		\begin{tabular}{l|ll}
	& $s_H$ & $s_L$ \\ \hline
	$k_H$ & 10    & 10    \\
	$k_L$ & 30    & 50   
\end{tabular}
				\caption*{Case I: $n(k_H) < n(k_L,s_H)$}
			\end{table}
		\end{column}
		\begin{column}{0.4\textwidth}
			\begin{table}
		\begin{tabular}{l|ll}
	& $s_H$ & $s_L$ \\ \hline
	$k_H$ & 20    & 10    \\
	$k_L$ & 20    & 50 
\end{tabular}
\caption*{Case II: $n(k_H) > n(k_L,s_H)$}
			\end{table}
		\end{column}
	\end{columns}
		\begin{itemize}
		\item Case I: The 20 $k_H$ workers all match with the $(k_L,s_H)$ workers and there are 10 $(k_L,s_H)$ types leftover who will match with $(k_L,s_L)$ types. 
		\begin{itemize}
			\item In the 20 $k_H,s_H$ matches, I can't say anything about the $k$-role's $s$ type.
		\end{itemize}
		\smallskip
		\item Case II: There will be 20 $k_H$ and $s_H$ matches, and the remaining $k_H$ workers will match with an $(k_L,s_L)$ individual.
		\begin{itemize}
			\item In the 20 $k_H,s_H$ matches, I can't say anything about the $k$-role's $s$ type.
		\end{itemize}
		\smallskip
		\item There will always be leftover $(k_L,s_L)$ types who self-match, but I don't know how they determine which worker is $k$ and which is $s$---assume this is random, I suppose.
	\end{itemize}
\end{frame}

\begin{frame}{Lack of identification in the $2\times 2$ model with matching}
	\begin{itemize}
		\item If matching doesn't introduce changes in occupational choice, it does not give us any additional information about the agents.
		\smallskip
		\item This outcome is identical to the case without matching since workers simply take prevailing wages as given, decide which role to choose, and then they are paired in firms to maximize output.
		\smallskip
		\item This is also true in the full version of the model with continuous types where the social planner solves an optimal transportation problem.
		\smallskip
		\item In fact, this is exactly how we solve the OT problem: Set wages such that half choose to be managers, half choose to be workers, and then solve the OT problem to get the matching.\footnote{Theorem 1 in the paper essentially says this.}

	\end{itemize}
\end{frame}
		\bibliography{/home/selliott/Research/bib/matching}
\end{document}
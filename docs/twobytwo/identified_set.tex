%%% Document class (e.g. article, letter, beamer, etc...)
\documentclass[12 pt]{article}
%%%

%%% Packages
\usepackage[round]{natbib} %Allows custom options for bibliography management
\usepackage{cite} %Adds bibliograhy functionality like \citet
\usepackage{geometry} %Allows me to set the margins
\usepackage{amsmath} %Adds additional math-related commands
\usepackage{amsfonts} %Adds fonts-within-math-mode functionality
\usepackage{amsthm} %Enables custom theorem environments
\usepackage{bm} %Adds bold math fonts
\usepackage{setspace} %Enables \doublespacing command
\usepackage{parskip} %Removes indentation
\usepackage{tikz} %Used for formatting plots and figures
\usepackage{float} %Used to force the position of tables and figures
\usepackage{graphicx} %For including images
\usepackage[T1]{fontenc} %Allows for encoding of slavic letters
\usepackage{lmodern} %Need to load back the original font for when T1 encoding is active
\usepackage{hyperref}
\usepackage{subcaption}
%%%

%%% Set the margins of the document
\geometry{
	left=1in,
	right=1in,
	bottom=1in,
	top=1in,
}
%%%

%%% NOTE: PLOT DIMENSIONS SHOULD BE AROUND 10 x 4

%%% User-defined commands

\DeclareMathOperator*{\argmax}{argmax} % The argmax operator

\newtheorem{assumption}{Assumption} % Treating assumptions like theorems.
\newtheorem{proposition}{Proposition} % Treating propositions like theorems.
\newtheorem{definition}{Definition}
\newtheorem{corollary}{Corollary}
\newtheorem{lemma}{Lemma}

% Footnote without superscripts --- used in title
\newcommand\blfootnote[1]{%
	\begingroup
	\renewcommand\thefootnote{}\footnote{#1}%
	\addtocounter{footnote}{-1}%
	\endgroup
}

%%%

%%% Title
\title{Identification in a $2\times2$ occupational and matching
	model}
%\author{Sean M. T. Elliott\blfootnote{University of Toronto, sean.elliott@mail.utoronto.ca}}
\date{}
%%%


\begin{document}
	
	\maketitle
	
	\onehalfspacing
	
	\vspace{-2cm}
	
	\section{Preliminaries}
	

Consider a labour market with two occupations, managers whose skills are high and low, $M=\{H,L\},$ and workers whose skills are also high and low, $\omega=\{h,l\}$. There is a measure 1 of labor market participants.Every participant has an managerial skill, $M$, and a worker skill, $\omega$. Every participant chooses the occupation which maximizes their income. The matrix distribution of occupational skills across participants is given below:%
\[
P(M,\omega)=%
\begin{array}
	[c]{ccc}
	& \omega=l & \omega=h\\
	M=L & p_{Ll} & p_{Lh}\\
	M=H & p_{Hl} & p_{Hh}%
\end{array}
\]
where
\begin{align}
	&  p_{Hh}+p_{hL}+p_{Lh}+p_{Ll}=1\\
	& p_{ij}   >0;i=L,H;j=l,h.
\end{align}
Each firm consists of a manager and a worker. They produce a unit of output to sell in the output market. The quality of the output, which is also the revenue of the firm, by a manager of skill $M$ and a worker of skill $\omega$ is $R(M,\omega\dot{)}$.
\begin{assumption}
	$R(M,\omega)$, revenue, is strictly
	increasing in the occupational skills of the firm.
	\label{a1}
\end{assumption}
\begin{assumption}
	$R(M,\omega)$, revenue, is max supermodular
	in occupational skills\footnote{When $R(i,j)$ is continuous in occupational
	skills, max supermodular is the same as supermodular.}	
	\begin{equation}
		R(H,h)+R(L,l)>2\max(R(H,l),R(L.h))
	\end{equation}
		\label{a2}
\end{assumption}
Let managers (owners of the firms) hire workers in a competitive labor market
for workers. Let the market earnings of a worker of skill $\omega$ be
$w(\omega)$. The manager of skill $M$ will choose a worker to solve:
\begin{equation}
	\pi^{\ast}(M,w(\omega))=\max_{\omega}R(M,\omega)-w(\omega) \label{profit max}
\end{equation}
$\omega^{\ast}(M,w(\omega))$, also known as the matching function, is the
worker who is optimally chosen by the manager:
\begin{equation}
	\omega^{\ast}(M)=\arg\max_{\omega}R(M,\omega)-w(\omega) \label{worker choice}
\end{equation}


As is well know, supermodularity implies that there will be positive
assortative matching by occupational skills within each firm.

Let $\pi^{\ast}(M,\omega^{\ast}(M))$ be the maximized profits of the manager
$M$ and $w(\omega^{\ast}(M))$ be the maximized earnings of the worker in the firm:%

\begin{equation}
	R(M,\omega^{\ast}(M\dot{)})=\pi^{\ast}(M\dot{)}+w(\omega^{\ast}(M\dot{)})
	\label{5}%
\end{equation}


If a worker's skill, $\omega$, is chosen by two types of managers, his wage
only depends on his skill $\omega$ and not which type of manager he is matched
to. We make a similar argument for managers. If a manager of skill $M$ is
indifferent between two types of workers, her maximized profits is $\pi^{\ast
}(M,\omega^{\ast}(M))$, independent of which type of worker she actually chooses.

Given their occupational skills, ($M,\omega)$, every participant will choose
the occupation which pays them the highest income:%

\begin{equation}
	O^{\ast}(M,\omega,\pi^{\ast},w^{\ast}),=\arg\max_{M,\omega}(\pi^{\ast
	}(M,\omega^{\ast}(M)),w^{\ast}(\omega)) \label{occuptional choice}%
\end{equation}


Due to the competitive labor market assumption, for every active firm in the
industry, its revenue, $R(M,\omega^{\ast}(M\dot{)})$, is equal to the
maximized profits and its wage bill%

\begin{equation}
	R(M,\omega^{\ast}(M\dot{)}\dot{)}=\pi^{\ast}(M,\omega^{\ast}(M\dot{)}\dot
	{)}+w^{\ast}(\omega^{\ast}(M\dot{)}\dot{)} \label{6}%
\end{equation}


We will now derive the demand and supply of occupational skills for the
industry. Let $I^{\ast}(M,\omega,\pi^{\ast},w^{\ast})$ be an indicator
function which takes the value 1 if a participant of skill ($M,\omega)$
strictly chooses to be a manager, $\pi^{\ast}(M,\omega^{\ast}(M))>w^{\ast
}(\omega)$, and is zero otherwise. Note that for any managerial skill $M$,
there is at most two types of participant $(M,\omega)$ for which $I^{\ast
}(M,\omega,\pi^{\ast},w^{\ast})>0$. Let $\phi(M,\omega,\pi^{\ast},w^{\ast}%
)\in(0,1]$ if a participant of skill ($M,\omega)$ is indifferent between being
a manager or a worker, $\pi^{\ast}(M,\omega^{\ast}(M))=w^{\ast}(\omega)$, and
is zero otherwise. Note that for any managerial skill $M$, there is only one
type of participant $(M,\omega)$ for which $\phi(M,\omega,\pi^{\ast},w^{\ast
})>0$.

The mass (market supply) of managers of skill $M$, $S^{\ast}(M,\pi^{\ast
},w^{\ast})$ is:%

\begin{equation}
	S^{\ast}(M,\pi^{\ast},w^{\ast})=\phi(M,\omega,\pi^{\ast},w^{\ast}%
	)p_{M_{\omega}}+\sum_{M,\omega}I^{\ast}(M,\omega,\pi^{\ast},w^{\ast
	})p_{M_{\omega}} \label{supply of managers}%
\end{equation}


Let $J^{\ast}(M,\omega)$ be an indicator function which takes the value 1 if a
manager of skill $M$ hires a worker of skill $\omega$ and zero otherwise. The
mass (market demand) for workers of skill $\omega$, $d^{\ast}(\omega,\pi
^{\ast},w^{\ast})$ is:%

\begin{equation}
	d^{\ast}(\omega,\pi^{\ast},w^{\ast})=\sum_{M}J^{\ast}(M,\omega)S^{\ast}%
	(M,\pi^{\ast},w^{\ast}) \label{demand for workers}%
\end{equation}


Let $i^{\ast}(M,\omega,\pi^{\ast},w^{\ast})$ be an indicator function which
takes the value 1 if a participant of skill $(M,\omega)$ strictly prefers to
be a worker and zero otherwise. Let $\Phi(M,\omega,\pi^{\ast},w^{\ast}%
)\in(0,1]$ if a participant of skill ($M,\omega)$ is indifferent between being
a manager or a worker, $\pi^{\ast}(M,\omega^{\ast}(M))=w^{\ast}(\omega)$, and
is zero otherwise. The mass (market supply) of workers of skill $\omega$,
$n^{\ast}(\omega,\pi^{\ast},w^{\ast})$ is:%

\begin{equation}
	n^{\ast}(\omega,\pi^{\ast},w^{\ast})=\Phi(M,\omega,\pi^{\ast},w^{\ast
	})p_{M_{\omega}}+\sum_{(M,w)}i^{\ast}(M,\omega,\pi^{\ast},w^{\ast
	})p_{M_{\omega}} \label{supply of workers}%
\end{equation}


We now define a competitive equilibrium for this labor market:

\begin{definition}
	\begin{enumerate}
		Let $P(M,w)$ be the population supply of skills.
		
		\item A competitive equilibrium is an equilibrium wage function, $w^{\ast}%
		(w)$, such that:
		
		\item All labor market participants choose the occupations which maximize
		their incomes. I.e. satisify (\ref{occpational choice}).
		
		\item All managers choose workers which maximizes their profits. I.e. satisfy
		(\ref{profit maximization}) and (\ref{worker choice}).
		
		\item The market demand for workers is equal to the market supply of workers
		by skills:%
		\begin{equation}
			d^{\ast}(\omega,\pi^{\ast},w^{\ast})=s^{\ast}(\omega,\pi^{\ast},w^{\ast
			}),\omega=\{h,l\} \label{market equilibrium}%
		\end{equation}
		
	\end{enumerate}
\end{definition}

\begin{description}
	\item[Assumption] \label{3}\bigskip\ $p_{Hh}<p_{Hl}<\,p_{Lh}<p_{Ll}%
	<p_{Hl}\,+p_{Lh}$
\end{description}

The population size by type of participants is decreasing in the skills of the
participants, consistent with commonly observed occupational skills
distributions. While $p_{Ll}$ is larger than $p_{Hl}$ or $p_{Lh}$, it is not
larger than the sum of all the mixed skills populations. This last restriction
restricts the population size of individuals with no occupational comparative advantage.
	
	

	
	\newpage
	\bibliographystyle{plainnat}
	\bibliography{/home/selliott/Research/bib/matching}
	
\end{document}

%%% Document class (e.g. article, letter, beamer, etc...)
\documentclass[12 pt]{article}
%%%

%%% Packages
\usepackage[round]{natbib} %Allows custom options for bibliography management
\usepackage{cite} %Adds bibliograhy functionality like \citet
\usepackage{geometry} %Allows me to set the margins
\usepackage{amsmath} %Adds additional math-related commands
\usepackage{amsfonts} %Adds fonts-within-math-mode functionality
\usepackage{amsthm} %Enables custom theorem environments
\usepackage{bm} %Adds bold math fonts
\usepackage{setspace} %Enables \doublespacing command
\usepackage{parskip} %Removes indentation
\usepackage{tikz} %Used for formatting plots and figures
\usepackage{float} %Used to force the position of tables and figures
\usepackage{graphicx} %For including images
\usepackage[T1]{fontenc} %Allows for encoding of slavic letters
\usepackage{lmodern} %Need to load back the original font for when T1 encoding is active
%%%

%%% Set the margins of the document
\geometry{
	left=1in,
	right=1in,
	bottom=1in,
	top=1in,
}
%%%

%%% User-defined commands

\DeclareMathOperator*{\argmax}{argmax} % The argmax operator

\newtheorem{assumption}{Assumption} % Treating assumptions like theorems.
\newtheorem{proposition}{Proposition} % Treating propositions like theorems.
\newtheorem{definition}{Definition} % Treating propositions like theorems.
\newtheorem{claim}{Claim}

%%%

%%% Title
\title{Identification in a Roy Model with Matching}
\author{}
\date{}
%%%

\begin{document}
\maketitle
\vspace*{-2cm}
\onehalfspacing

\section{Overview --- REWRITE after}

In this document I will outline a Roy model where workers match with one another in firms in a perfectly competitive environment. First, I will outline the classical \citet{roy1951some} model and the identification problem. After the model has been introduced, I will discuss what can be learned empirically (i.e, identified) about the unconditional distribution of worker skills and potential earnings from data on equilibrium wages and who matches with whom. I show that without additional assumptions, the well-known identification problem resulting from workers selecting into occupations [e.g. \citet{heckman1990empirical}, \citet{french2011identification}] persists even once we introduce matching to the classical Roy model \`{a} la \citet{mak2025occupational}.\footnote{See \citet{heckman1990empirical}, \citet{bayer2011nonparametric}, \citet{buera2006non}, or \citet{mourifie2020sharp} for cases where additional assumptions are made to obtain non-parametric identification.} I will then discuss some preliminaries ideas regarding what can be done to address this issue.  

\section{Some preliminaries}

Here I'll consider a model of occupational choice which is augmented to allow for workers match in firms as pairs. A competitive equilibrium in this environment are a pair of earnings schedules---one for each occupation---such that, given a distribution of workers and a firm production function, workers choose the occupation which maximizes their wage, firms choose workers to maximize profits, and the labour market clears. As such, before I expound the model, it is worth making a distinction between worker skills and wages. In the classical Roy model \citep{heckman1990empirical}, skills and log wages can be treated as equivalent. However, introducing a firm and workers matching, skills and wages cannot be replaced with one another without an appropriate transformation. Additionally, the wage functions are unknown ex ante and determined in equilibrium.

Assume that each worker's skills are drawn from an exogenously given distribution $F\in \Delta(\Theta)$ where $\Theta \subseteq \mathbb{R}_+^2$ and $\Delta(X)$ denotes the set of continuous probability distributions with support on $X$. In the classical Roy model $F$ is assumed to be the log-normal distribution and the wage functions for each occupation are assumed to be known. Suppose that an individual $i$ can choose between occupation $k$ or occupation $s$ where they earn $\pi_i$ or $w_i$, respectively. Let 
\begin{equation}
	J_i =
	\begin{cases}
		k \text{ if } \pi_i \geq w_i\\
		s \text{ if } \pi_i < w_i
	\end{cases}
	\label{eqn:occ_choice}
\end{equation}
represent the individual's occupational-choice problem. I'll define an indicator variable as 
\begin{equation*}
	D_i =
	\begin{cases}
		1 \text{ if } J_i=k\\
		0 \text{ if } J_i=s
	\end{cases}
\end{equation*}
denoting whether the individual chooses role $k$ or $s$. The observed wage for individual $i$ can be written using a switching equation as
\begin{equation*}
	Y_i = \pi_i D_i + (1-D_i) w_i
\end{equation*}
where both $\pi_i$ and $w_i$ are assumed to have the form
\begin{equation}
	\pi_i = \mu_\pi + \varepsilon_{\pi_i}
	\label{eqn:pi_simple}
\end{equation}
\begin{equation}
	w_i = \mu_w + \varepsilon_{w_i}
	\label{eqn:w_simple}
\end{equation}
with $\begin{bmatrix}
	\varepsilon_{\pi_i}\\
	 \varepsilon_{w_i}
\end{bmatrix} \sim N \begin{pmatrix}
\begin{bmatrix}
0\\
0
\end{bmatrix},
\begin{bmatrix}
	\sigma_\pi^2 & \sigma_{\pi w}\\
	\sigma_{\pi w} & \sigma_w^2 
\end{bmatrix}
\end{pmatrix}$. If, for each $i=1,\cdots, N$, we observe $(Y_i,D_i)$ then, using properties of normal random variables, \citet{heckman1990empirical} show that it is possible to uncover $\mu_\pi$, $\mu_w$, $\sigma_\pi^2$, $\sigma_w^2$, and $\sigma_{\pi w}$.\footnote{I show how this works in the Appendix.}

There are two interesting things to note about the model I have described thus far. The first thing is that potential earnings $(\pi_i,w_i)$ only depend on the exogenously-given distribution of worker skills. The second point of note is that if we exclude the assumption of normally distributed skills, then the distribution of potential earnings is no longer identified from the data $(Y_i,D_i)_{i=1}^N$. In fact, as \citet{heckman1990empirical} explain, given a cross-section of data on earnings and occupational choice, an econometrician would be unable to distinguish a model with positively (or negatively) correlated skills from one in which skills are independent. A key distinction between the classical Roy model and the Roy model with matching is that potential earnings $(\pi_i,w_i)$ are endogenous objects determined in equilibrium. However, as \citet{heckman1985heterogeneity} observe, endogenizing wages isn't enough to obtain identification of the unconditional skill distribution. To solve this problem, they assume skills are normally distributed. The goal in the analysis presented here will be to introduce matching to the classical Roy model and see if it is possible to identify the unconditional distribution of potential earnings without any parametric assumptions on the distribution of skills.

\section{Roy model with matching}

The occupational choice problem is the same as that which was described in the previous section. That is, workers choose either role $k$ or role $s$ based on which one provides them a higher wage. However, one key difference is we can no longer assume that wages are given by Equations (\ref{eqn:pi_simple}) and (\ref{eqn:w_simple}). Furthermore, we no longer assume that the distribution of skills $F$ is normal.

Consider a firm which employs exactly one $k$-type worker and exactly one $s$-type worker with a production function $R:\Theta \to \mathbb{R}_+$. Each firm will choose a pair of workers according to
\begin{equation}
	\max_{k^*,s^*} R(k^*,s^*) - \pi(k^*) - w(s^*)
	\label{eqn:firm_prob}
\end{equation}
where free entry implies that profits are zero.

\begin{definition}{Competitive equilibrium}
	
	
 Given a distribution of skills $F\in \Delta(\Theta)$ and a revenue function $R: \Theta \to \mathbb{R}_+$, a competitive equilibrium is a set of earning's schedules $\pi : \mathbb{R}_+ \to \mathbb{R}_+$ and $w:\mathbb{R}_+ \to \mathbb{R}_+$ such that
 \begin{enumerate}
 	\item Workers choose their occupations according to Equation (\ref{eqn:occ_choice}).
 	\item Firms choose workers according to the profit maximization condition in Equation (\ref{eqn:firm_prob}).
 	\item The labour market clears. This means that the endogenous supply of $k$-type workers is equal to the demand for $k$-type workers and the endogenous supply of $s$-type workers is equal to the demand for $s$-type workers.
 \end{enumerate}
\end{definition}
An implication of the market clearing mechanism and the fact that firms employ one worker of each type is that the earnings schedule must be such that exactly one half of workers choose the $k$ role and exactly one half choose the $s$ role. This places a restriction on the distribution of potential earnings which does not exist in the classical Roy model. 

\subsection{Earnings in the model}

Here I will describe how potential earnings are determined endogenously in the model. As in \citet{mak2025occupational}, the competitive equilibrium is implemented by a social planner who seeks to minimize the social cost of employing a worker of type $(k^*,s^*)$. This is the dual of the revenue maximization problem and we know from the theory of optimal transportation [see e.g. \citet{mccann2010optimal} or \citet{galichon2018optimal}] that the solution to the primal problem is equal to that of the dual. Let $\psi:\Theta \to \mathbb{R}$ be the social cost function. We can write the social planner's dual problem as
\begin{equation*}
	\min_{\tilde{\psi},\tilde{\pi},\tilde{w}} \int_0^\infty \int_0^\infty \tilde{\psi}(k^*,s^*) dF(k^*,s^*)
\end{equation*}
such that
\begin{align*}
	\tilde{\pi}(k^*) + \tilde{w}(s^*) &\leq R(k^*,s^*),\\
	\tilde{\pi}(k^*) &\leq \tilde{\psi}(k^*,s^*), \text{ and}\\
   \tilde{w}(s^*) &\leq\tilde{\psi}(k^*,s^*).
\end{align*}

Another way to think of the problem is by using Proposition 4 in \citet{mak2025occupational}. Given the worker's occupational choice problem, the social cost of hiring a worker of type $(k^*,s^*)$ is simply $\psi(k^*,s^*) = \max\{\tilde{\pi}(k^*),\tilde{w}(s^*)\}$. Thus, we can rewrite the objective function of dual problem as
\begin{equation}
	\min_{\tilde{\pi},\tilde{w}} \int_0^\infty \int_0^\infty \max\{\tilde{\pi}(k^*),\tilde{w}(s^*)\} dF(k^*,s^*).
	\label{eqn:spp}
\end{equation}
As such, it is clear that for an individual $i$, their potential earnings $(\pi_i,w_i)$ depend not only on their skill level $(k_i,s_i)$ but also on the solution to the optimization problem the social planner faces. This is in contrast to the model in Section 2 where an individual's potential earnings depends only on their randomly-drawn skill level given in Equations (\ref{eqn:pi_simple}) and (\ref{eqn:w_simple}).

Let $E \in \Delta(\Theta)$ be the distribution of potential earnings in the population. Proposition 5 in \citet{mak2025occupational} tells us that if we observe a mass of firms where the $k$-type worker earns $\pi^*$ and the $s$-type worker earns $w^*$, it must be that
\begin{equation}
	\int_0^{\pi^*} E(\pi^*,\tilde{w}) d\tilde{w} = \int_0^{w^*} E(\tilde{\pi},w^*) d \tilde{\pi}.
	\label{eqn:prop5}
\end{equation}
Intuitively, this is because each firm employs exactly two workers (bilateral matching) and each of these workers chooses their role optimally (occupational choice). While this fact is interesting, it doesn't depend on the fact that wages are determined endogenously. In fact, Equation (\ref{eqn:prop5}) would hold even if the matching was random and wages were exogenously given so long as the matching was bilateral and workers chose occupations according to Equation (\ref{eqn:occ_choice}).\footnote{Suppose Equation (\ref{eqn:prop5}) doesn't hold and there is a mass of firms whose employees earn $(\pi^*,w^*)$ where $\int_0^{\pi^*} E(\pi^*,\tilde{w}) d\tilde{w} > \int_0^{w^*} E(\tilde{\pi},w^*) d \tilde{\pi}$. Since the quantity on the LHS is increasing in the upper limit of integration, this implies we can find an $\epsilon > 0$ such that $\int_0^{\pi^*-\epsilon} E(\pi^*,\tilde{w}) d\tilde{w} = \int_0^{w^*} E(\tilde{\pi},w^*) d \tilde{\pi}$. Thus, since the matching is bilateral, there must be a mass $\int_{\pi^*-\epsilon}^{\pi^*} E(\pi^*,\tilde{w}) d\tilde{w}$ of unmatched workers, which is a contradiction. } In that sense, this result alone doesn't provide anything useful when trying to solve the identification problem. However, what is useful to us is that, in the case where a social planner implements the competitive equilibrium, wages must be such that the market clears. I'm going to illustrate this why this is helpful using an example with a simplified 2$\times$2 skill space.

\subsection{The 2$\times$2 model}

Consider a setting where skills are discrete and the skill space is $\Theta = K \times S$ with $K=\{k_L,k_H\}$ and $S=\{s_L,s_H\}$. Each worker $i=1,\cdots,N$ is randomly endowed with an ordered pair of skills $(k_i,s_i)\in\Theta$. I'll consider a series of different forms of the model to illustrate how the identification problem changes as we introduce more structure. First, I'll consider the classical Roy model in this simplified 2$\times$2 setting. Next, I'll consider what happens when we introduce random matching to the occupational choice problem. Finally, I'll consider what happens when we introduce a firm that tries to match workers optimally. In this case, I'll consider two separate scenarios. In the first scenario, firms will choose workers to maximize output but they face a fixed wage structure meaning they cannot influence occupational choice. In the second scenario, the wage is free to adjust. I will show that going from scenario one to scenario two gives us some identification power.

\subsubsection{Roy model without matching}

This is the classical Roy model where the distribution of skills is given exogenously and potential earnings depend only on the skill level of the worker. The only thing I will impose is a monotonicity assumption on the earnings. This is just to focus the problem---a different ordering would produce analogous results.

\begin{assumption}(Monotonicity)
	Without loss of generality, suppose that the wage functions $\pi:K\to\mathbb{R}_+$ and $w:S\to\mathbb{R}_+$ are fixed such that $\pi(k_H) > w(s_H) > \pi(k_L) > w(s_L)$.
	\label{ass:mono}
\end{assumption}

For now, we will shut down matching and simply assume that workers choose their occupation to maximize their wage. We can establish a basic non-identification result in this setting.

\begin{proposition}{Lack of identification---2$\times$2 without matching}
	
	In the 2$\times$2 setting described above, we cannot fully identify the underlying distribution of skills.
	\label{prop:identnomatch}
\end{proposition}
\begin{proof}
\begin{table}[H]
	\centering
	\begin{tabular}{l|ll}
		& $s_H$        & $s_L$        \\ \hline
		$k_H$ & $n(k_H,s_H)$ & $n(k_H,s_L)$ \\
		$k_L$ & $n(k_L,s_H)$ & $n(k_L,s_L)$
		
	\end{tabular}
	\caption{Distribution of skills}
	\label{tab:2by2counts}
\end{table}
Let $n(k,s)$ be the count of type $(k,s)\in \Theta$ individuals in the population---this is the object we want to uncover. Equipped with Assumption \ref{ass:mono} and the ability to observe occupational choice and realized earnings, we can say the following:
\begin{enumerate}
	\item If an individual chooses $s$ and is paid $w(s_H)$, their $k$ type must be $k_L$ since if it were $k_H$ they would have chosen the $k$ role instead. This implies that $n(k_L,s_H)$ is identified from the number of $s_H$ types observed in the population.
	\item If a worker chooses $k$ and is paid $\pi(k_L)$, they are necessarily also type $s_L$ for the same reason as above. This implies $n(k_L,s_L)$ is identified from the number of $k_L$ workers in the population. Thus we have the bottom row of Table \ref{tab:2by2counts}.
	\item No worker will choose $s_L$ since it is the worst option regardless of a worker's $k$ type.
	\item If $k$ is chosen and they earn $\pi(k_H)$, this implies they could be either $s_L$ or $s_H$ since $k_H$ is strictly preferred to both. As such we know the sum of the top row of Table \ref{tab:2by2counts} $n(k_H)=\sum_{\tilde{s}\in S}n(k_H,\tilde{s})$ but we cannot identify the counts separately.
\end{enumerate}
\end{proof}
The result above is a special case of the well-known Roy model identification problem which stems from workers selecting into the role in which they earn the most. Even with an extremely simplified skill space, the problem persists. 

\subsubsection{Roy model with random matching}

For now, suppose that there is no attempt to match workers in a way which maximizes output. Instead, each worker $i$ chooses their occupations according to $\max\{\pi_i,w_i\}$ and then they are paired randomly. Two workers can only be paired with one another if they are opposite types. From Proposition \ref{prop:identnomatch} we know that the number of $k$-type workers is $n(k) = n(k_H,s_H) + n(k_H,s_L) + n(k_L,s_H)$. Similarly, we also know that the number of $s$-type workers is $n(s)=n(k_L,s_H)$. In this case, we are almost certainly going to have unpaired workers. This is because the market clears if and only if $n(k) = n(s)$. However, given the strict monotonicity of wages (Assumption \ref{ass:mono}) and occupational choice there is no guarantee that this will occur. Instead, if we suppose that the distribution of skills is roughly uniform, it is far more likely to be that $n(k)>n(s)$. The purpose of introducing this case is to motivate why it is useful to introduce matching with firms that optimize. Without wage schedules determined in equilibrium, we are stuck in the classical Roy model world where without additional assumptions, self-selection prevents us from identifying the skill distribution. I'm going to go through this in the next section where first I consider firms which choose workers optimally but wages are fixed and then I'll introduce a mechanism to clear the market.

\subsubsection{Roy model with optimal matching and fixed wages}

Now I'll introduce a firm which chooses workers optimally. I'm going to make a few assumptions about the firm's revenue function and the distribution of workers.

\begin{assumption}(Revenue function)
	The firm's revenue function $R:\Theta\to\mathbb{R}_+$ is strictly supermodular (i.e., $R(k_H,s_H) + R(k_L,s_L) > R(k_L,s_H) + R(k_H,s_L)$) which will induce positive assortative matching (PAM). 
	\label{ass:rev}
\end{assumption}

\begin{assumption}(Unmatched workers)
	Workers can only produce output in pairs organized in firms. That is, if a worker is unmatched, they produce and earn nothing.
	\label{ass:unmatched}
\end{assumption}

\begin{assumption}
	(Talent scarcity) High-skill $k$ types are more scarce than high-skill $s$ types.  That is, the distribution of skills in the population is such that
	\begin{equation*}
		n(k_H) < n(s_H).
	\end{equation*}
	\label{ass:scarcity}
\end{assumption}

\begin{assumption}
	(Positive correlation) If a worker is high/low skill in one occupation they are also more likely to be high/low skill in the other implying that
	\begin{equation*}
		\sum_{\tau\in\{L,H\}}n(k_\tau,s_\tau) > \sum_{\tau\in\{L,H\}}n(k_\tau,k_{\sim \tau})
	\end{equation*}
	where $\sim \tau = L$ if $\tau = H$ and vice versa. 
	\label{ass:correl}
\end{assumption}

For the sake of illustration, first suppose that wages are fixed according to the order given in Assumption \ref{ass:mono}. The purpose of this is to show that matching alone doesn't provide any identification power. This result is summarized in the following proposition.
\begin{proposition}{Lack of identification---2$\times$2 with matching and fixed wages}
	
	Under Assumptions \ref{ass:mono}--\ref{ass:correl} we cannot identify the distribution of skills from data on occupational choice, earnings, and who is matched with whom. 
	\label{prop:noidentmatch}
\end{proposition}
\begin{proof}
	Due to PAM (Assumption \ref{ass:rev}), firms will want to match $k_H$ types with $s_H$ types workers and $k_L$ types with $s_L$ types. However, from Assumption \ref{ass:scarcity} we know that there are fewer $k_H$ types than $s_H$ types implying that we can't simply pair these workers up. Furthermore, under the prevailing wage structure stated in Assumption \ref{ass:mono}, a worker who is $k_H$ will always choose the $k$ role. That is, the supply of $s_H$ workers is only those who are also type $k_L$ as established in Proposition \ref{prop:identnomatch}. Therefore, we can consider the three exhaustive cases regarding $n(k_H)$ and I'll show that the non-identification result established earlier continues to hold. 
	\begin{itemize}
		\item Case I: $n(k_H) < n(k_L,s_H)$
		
		In this scenario, all of the $k_H$ workers will match with $(k_L,s_H)$ types. There are leftover $(k_L,s_H)$ types who will match with $(k_L,s_L)$ types as these individuals will choose $k_L$ due to Assumption \ref{ass:mono}. Let $\mu(k^*,s^*)$ denote the observed number of matches between a type $k^*$ and a type $s^*$ individual. We know that $\mu(k_L,s_H) = n(k_L,s_H)-n(k_H)$ which gives us $n(k_L,s_H)$ since we know that $\mu(k_H,s_H) = n(k_H)$. However, we also know that $n(k_H) = n(k_H,s_H) + n(k_H,s_L)$ implying that within the set of $(k_H,s_H)$ matches, we are unable to infer the $k$-type worker's $s$ type. We can also figure out $n(k_L,s_L)$ from $\mu(k_L,s_H)$ and the number of individuals who are unmatched. 
		
		\item Case II: $n(k_H) > n(k_L,s_H)$
		
		Now all of the $n(k_L,s_H)$ workers will match with $n(k_H)$ workers, but instead we will have an excess of $k_H$ types. These workers cannot match with $(k_L,s_L)$ types since they choose the $k$ role since $\pi(k_L)>w(s_L)$. As before, we know $\mu(k_H,s_H)=n(k_L,s_H)$. However, in this case there are unmatched $k_H$ workers who we would  not be able to differentiate from the unmatched $(k_L,s_L)$ workers due to Assumption \ref{ass:unmatched}.
		
		\item Case III: $n(k_H) = n(k_L,s_H)$
		
		Finally, in this case $\mu(k_H,s_H) = n(k_H) = n(k_L,s_H)$ and again, we don't know anything about the $k$-type worker's $s$ type. All of the $(k_L,s_L)$ types remain unmatched since they all choose to be the $k$ role. Again, all we can identify is $n(k_H)$, $n(k_L,s_H)$, and $n(k_L,s_L)$.
	\end{itemize}
	
	Thus, in each case we have an similar result to Proposition \ref{prop:identnomatch}. We can identify $n(k_L,s_H)$ and $n(k_L,s_L)$ but all we know about the remaining workers is that $n(k_H)=\sum_{\tilde{s}\in S}n(k_H,\tilde{s})$
	
\end{proof}

Proposition \ref{prop:noidentmatch} highlights what happens if we do not consider a mechanism which allows wages to adjust to clear the market. Of course, it leads to unrealistic results which would not arise in reality. For example, in all three cases there are leftover $(k_L,s_L)$ workers. Since we assume that that unmatched workers produce nothing (Assumption \ref{ass:unmatched}), this is inefficient in the sense that total output could increase if we could entice some of these workers to switch from choosing the $k$ role to the $s$ role. One way to do this is to set $\pi(k_L) = w(s_L)$. In this case, $(k_L,s_L)$ workers would be indifferent between the two roles and we could pair them up and increase total output. Furthermore, as I'll show in the next section, we can also use this idea to say something about the identification of skills. 

\subsubsection{Roy model with optimal matching and wage adjustments}
\begin{table}[H]
	\centering
	\begin{tabular}{l|ll}
		& $s_H$        & $s_L$        \\ \hline
		$k_H$ & 20 & 10 \\
		$k_L$ & 20 & 50
		
	\end{tabular}
	\caption{Counts of skill types---numeric example}
	\label{tab:2by2countsex}
\end{table}
For the sake of illustration I'm going to describe how this works with a numeric example. In particular, I'll consider the case from Proposition \ref{prop:noidentmatch} where $n(k_H) > n(k_L,s_H)$ and the counts satisfy Assumptions \ref{ass:scarcity} and \ref{ass:correl}. The example I'll focus on is shown in Table \ref{tab:2by2countsex}. Consider two time periods where in each period we can observe occupational choice, earnings, and who is matched with whom. That is, we have a panel of workers who we observe in both periods. In the first period, suppose that wages are fixed according to Assumption \ref{ass:mono}, and workers are paired together as described in Proposition \ref{prop:noidentmatch}. In the second period, we will allow wages to adjust ensure that all workers are employed and total output is maximized. Given Assumption \ref{ass:rev}, this implies that we want to match $k_H$ types with $s_H$ types and $k_L$ types with $s_L$ types. Let $\mu_t(k,s)$ denote the number of $(k,s)$ matches in period $t=1,2$. 

\begin{proposition}
	In the two-period case with $n(k_H) > n(k_L,s_H)$ as described above, we can partially identify the $s$ type of the $k_H$ workers.
\end{proposition}
\begin{proof}
	I'll go through each period separately to describe how the matching works and then I'll show how this provides partial identification of the skill distribution. 
	\begin{enumerate}
		\item In the first period where wages are fixed according to $\pi(k_H) > w(s_H) > \pi(k_L) > w(s_L)$, those who have $k$ type $k_H$ will prefer the $k$ role regardless of their $s$ type implying $n(k_H)=30$. These workers will then match with $(k_L,s_H)$ types who choose the $s$ role. Since $n(k_L,s_H)=20$ this implies we will have $\mu_1(k_H,s_H) = 20$. The remaining 10 $k_H$ workers with type will not match with $(k_L,s_L)$ workers since they also choose the $k$ role. That is, all $(k_L,s_L)$  types remain unmatched and 10 of the $k_H$ are unmatched. 
		
		This is not optimal for two reasons. The first is that we could increase $\mu_1(k_H,s_H)$ if we could incentivize some of the $(k_H,s_H)$ workers to choose $s$ instead of $k$. The second reason is that the leftover  $(k_L,s_L)$ types produce nothing.
		
		\item In the second period, the wages will adjust to correct for the two inefficiencies occurring in period 1. One way to do this is to set $\pi(k_H)=w(s_H)$ and $\pi(k_L)=w(s_L)$. Now occupational choice is such that, if you're high or low type of both, you're indifferent between $k$ and $s$. However, if you're high type of one and low type of the other, you prefer the occupation in which you are a high type.
		
		 First let's allocate the $k_H$ workers. We can assign the $n(k_H,s_L)=10$ workers to 10 of the $(k_L,s_H)$ workers to generate 10 $(k_H,s_H)$ matches. Since $n(k_L,s_H)=20$, we can take the remaining 10 and assign them to 10 of the $n(k_H,s_H)=20$ workers to generate 10 more $(k_H,s_H)$ matches. Now we have 10 leftover $(k_H,s_H)$ workers who are indifferent between $k$ and $s$ who we can pair them up implying $\mu_2(k_H,s_H) = 25$. Thus we find that  $\mu_2(k_H,s_H) > \mu_1(k_H,s_H)$. The remaining workers are $(k_L,s_L)$ types and are indifferent between the two roles, so we pair them up and get $\mu_2(k_L,s_L)=25$.  Since $\mu_2(k_L,s_L)=0$, this is clearly an improvement.
		 
		 What does this imply regarding identification? We know that in the $\mu_2(k_L,s_L)-\mu_1(k_H,s_H)=5$ new matches, these individuals are necessarily $(k_H,s_H)$ type. Thus we know that $n(k_H,s_H) \geq 10$. We can also infer that there are $n(k_H)=30$ in total and since $n(k_H) = n(k_H,s_H) + n(k_H,s_L)$ it must be that $n(k_H,s_L)\leq 20$. Thus we have bounds for the counts of the $s$ types of the $k_H$ workers. We know that $10 \leq n(k_H,s_H) \leq 30$ and $0 \leq n(k_H,s_L)\leq 20$. Also, since $\mu_2(k_H,s_H)>\mu_1(k_H,s_H)$ we can tell that $n(k_H) > n(k_L,s_H)$ implying we necessarily know that $n(k_L,s_H) = \mu_1(k_H,s_H) = 20$.  Furthermore, from the $\mu_2(k_L,s_L)=25$ we know that $n(k_L,s_L)=50$.  
	\end{enumerate}
\end{proof}

\section{Next steps}


\newpage
\bibliographystyle{plainnat}
\bibliography{/home/selliott/Research/bib/matching}

\newpage
\setcounter{section}{0}
\renewcommand{\thesection}{\Alph{section}}
\counterwithin*{equation}{section}
\renewcommand\theequation{\thesection\arabic{equation}}
\section{Appendix}
\subsection{Roy model identification with normally distributed skills}

In this section I will describe the simple case where we assume that skills are normally distributed. See either \citet{heckman1990empirical} or \citet{french2011identification} for more details. In this case wages are given by
\begin{equation}
	\pi_i = \mu_\pi + \varepsilon_{\pi_i}
	\label{eqn:pi_simple_app}
\end{equation}
\begin{equation}
	w_i = \mu_w + \varepsilon_{w_i}
	\label{eqn:w_simple_app}
\end{equation}
with $\begin{bmatrix}
	\varepsilon_{\pi_i}\\
	\varepsilon_{w_i}
\end{bmatrix} \sim N \begin{pmatrix}
	\begin{bmatrix}
		0\\
		0
	\end{bmatrix},
	\begin{bmatrix}
		\sigma_\pi^2 & \sigma_{\pi w}\\
		\sigma_{\pi w} & \sigma_w^2 
	\end{bmatrix}
\end{pmatrix}$.

Let
\begin{equation*}
	\lambda(\cdot) = \frac{\phi(\cdot)}{\Phi(\cdot)}
\end{equation*}
where $\phi$ and $\Phi$ are the standard normal pdf and cdf, respectively. Define 
	\begin{equation*}
	c = \frac{\mu_\pi - \mu_w}{\sqrt{\sigma_\pi^2 + \sigma_w^2 - 2\sigma_{\pi w}}}
\end{equation*} and for each $j\in\{\pi,w\}$ define
\begin{equation*}
	\kappa_j = \frac{\sigma^2_j-\sigma_{\pi w}}{\sqrt{\sigma_\pi^2+\sigma_w^2-2\sigma_{\pi w}}}.
\end{equation*}
We know from \citet{heckman1990empirical} the following properties of normal random variables:
\begin{equation}
	Pr(D_i = 1) = \Phi(c)
	\label{eqn:appoccchoice}
\end{equation}
\begin{equation}
	E(Y_i | D_i = 1) = \mu_\pi + \kappa_\pi \lambda(c)
	\label{eqn:meand1}
\end{equation}
\begin{equation}
	E(Y_i | D_i = 0) = \mu_w + \kappa_w \lambda(-c)
	\label{eqn:meand0}
\end{equation}
\begin{equation}
	Var(Y_i | D_i = 1) = \sigma_\pi^2 + \kappa_\pi^2(-\lambda(c)c - \lambda^2(c))
	\label{eqn:vard1}
\end{equation}
\begin{equation}
	Var(Y_i | D_i = 0) = \sigma_w^2 + \kappa_w^2(\lambda(-c)c - \lambda^2(c))
	\label{eqn:vard0}
\end{equation}
\begin{equation}
	E([Y_i - E(Y_i|D_i = 1)]^3|D_i = 1) = \kappa_\pi^3 \lambda(c)[2\lambda^2(c) + 3 c \lambda(c) + c^2 -1]
	\label{eqn:skewd1}
\end{equation}
\begin{equation}
	E([Y_i - E(Y_i|D_i = 0)]^3|D_i = 0) = \kappa_w^3 \lambda(-c)[2\lambda^2(-c) - 3 c \lambda(-c) + c^2 -1].
	\label{eqn:skewd0}
\end{equation}

We want to use these equations to solve for $\mu_\pi$, $\mu_w$, $\sigma_\pi^2$, $\sigma_w^2$, and $\sigma_{\pi w}$. First note that everything on the LHS of Equations (\ref{eqn:appoccchoice})--(\ref{eqn:skewd0}) is observed. Thus, $c=\Phi^{-1}(Pr(D_i = 1))$ is known which implies that $\lambda(c)$ and $\lambda(-c)$ are known. Thus, immediately we can get $\kappa_\pi^3$ and $ \kappa_w^3$ from Equations (\ref{eqn:skewd1}) and (\ref{eqn:skewd0}) and now the remaining equations (\ref{eqn:meand1}--\ref{eqn:vard0}) give us $\mu_\pi$, $\mu_w$, $\sigma_\pi^2$, and $\sigma_w^2$. Plug these into the expression for $c$ to solve for $\sigma_{\pi w}$. $\qed$

\end{document}

%%% Document class (e.g. article, letter, beamer, etc...)
\documentclass[12 pt]{article}
%%%

%%% Packages
\usepackage[round]{natbib} %Allows custom options for bibliography management
\usepackage{cite} %Adds bibliograhy functionality like \citet
\usepackage{geometry} %Allows me to set the margins
\usepackage{amsmath} %Adds additional math-related commands
\usepackage{amsfonts} %Adds fonts-within-math-mode functionality
\usepackage{amsthm} %Enables custom theorem environments
\usepackage{bm} %Adds bold math fonts
\usepackage{setspace} %Enables \doublespacing command
\usepackage{parskip} %Removes indentation
\usepackage{tikz} %Used for formatting plots and figures
\usepackage{float} %Used to force the position of tables and figures
\usepackage{graphicx} %For including images
\usepackage[T1]{fontenc} %Allows for encoding of slavic letters
\usepackage{lmodern} %Need to load back the original font for when T1 encoding is active
%%%

%%% Set the margins of the document
\geometry{
	left=1in,
	right=1in,
	bottom=1in,
	top=1in,
}
%%%

%%% User-defined commands

\DeclareMathOperator*{\argmax}{argmax} % The argmax operator

\newtheorem{assumption}{Assumption} % Treating assumptions like theorems.
\newtheorem{proposition}{Proposition} % Treating propositions like theorems.
\newtheorem{definition}{Definition} % Treating propositions like theorems.

%%%

%%% Title
\title{Identification in a Roy Model with Matching}
\author{}
\date{}
%%%

\begin{document}
\maketitle
\vspace*{-2cm}
\onehalfspacing

\section{Overview --- REWRITE after}

In this document I will outline a Roy model where workers match with one another in firms in a perfectly competitive environment. First, I will outline the basic Roy model and the identification problem. After the model has been introduced, I will discuss what can be learned empirically (i.e, identified) about the unconditional distribution of worker skills and potential earnings from data on equilibrium wages and who matches with whom. I show that without additional assumptions, the well-known identification problem resulting from workers selecting into occupations [e.g. \citet{heckman1990empirical}, \citet{french2011identification}] persists even once we introduce matching to the classical Roy model \`{a} la \citet{mak2025occupational}.\footnote{See \citet{heckman1990empirical}, \citet{bayer2011nonparametric}, \citet{buera2006non}, or \citet{mourifie2020sharp} for cases where additional assumptions are made to obtain non-parametric identification.} I will then discuss some preliminaries ideas regarding what can be done to address this issue.  

\section{Some preliminaries}

Here I'll consider a model of occupational choice \citep{roy1951some} which is augmented to allow for workers match in firms as pairs. A competitive equilibrium in this environment are a pair of earnings schedules---one for each occupation---such that, given a labour force and a firm production function, workers choose the occupation which maximizes their wage, firms choose workers to maximize profits, and the labour market clears. As such, before I expound the model, it is worth making a distinction between worker skills and wages. In the classical Roy model \citep{heckman1990empirical}, skills and log wages can be treated as equivalent. However, in this environment, there is a mapping between skills and wages, but they cannot be replaced with one another without an appropriate transformation. 

Assume that each worker's skills are drawn from an exogenously given distribution $F\in \Delta(\Theta)$ where $\Theta \subseteq \mathbb{R}_+^2$ and $\Delta(X)$ denotes the set of continuous probability distributions with support on $X$. In the classical Roy model $F$ is assumed to be the log-normal distribution and the wage functions for each occupation are assumed to be known. Suppose that an individual $i$ can choose between occupation $k$ or occupation $s$ where they earn $\pi_i$ or $w_i$, respectively. Let 
\begin{equation}
	J_i =
	\begin{cases}
		k \text{ if } \pi_i \geq w_i\\
		s \text{ if } \pi_i < w_i
	\end{cases}
	\label{eqn:occ_choice}
\end{equation}
represent the individual's occupational-choice problem. I'll define an indicator variable as 
\begin{equation*}
	D_i =
	\begin{cases}
		1 \text{ if } J_i=k\\
		0 \text{ if } J_i=s
	\end{cases}
\end{equation*}
denoting whether the individual chooses role $k$ or $s$. The observed wage for individual $i$ can be written using a switching equation as
\begin{equation*}
	Y_i = \pi_i D_i + (1-D_i) w_i
\end{equation*}
where both $\pi_i$ and $w_i$ are assumed to have the form
\begin{equation}
	\pi_i = \mu_\pi + \varepsilon_{\pi_i}
	\label{eqn:pi_simple}
\end{equation}
\begin{equation}
	w_i = \mu_w + \varepsilon_{w_i}
	\label{eqn:w_simple}
\end{equation}
with $\begin{bmatrix}
	\varepsilon_{\pi_i}\\
	 \varepsilon_{w_i}
\end{bmatrix} \sim N \begin{pmatrix}
\begin{bmatrix}
0\\
0
\end{bmatrix},
\begin{bmatrix}
	\sigma_\pi^2 & \sigma_{\pi w}\\
	\sigma_{\pi w} & \sigma_w^2 
\end{bmatrix}
\end{pmatrix}$. If, for each $i$, we observe $(Y_i,D_i)$ then, using properties of normal random variables, \citet{heckman1990empirical} show that it is possible to uncover $\mu_\pi$, $\mu_w$, $\sigma_\pi^2$ $\sigma_{\pi w}$, $\sigma_{\pi w}$, and $\sigma_w^2$.\footnote{I included this derivation in the Appendix.}

There are two interesting things to note about the model I have described thus far. The first thing is that potential earnings $E_i = (\pi_i,w_i)$ depend exclusively on the exogenously-given distribution of worker skills. The second point of note is that if we drop the assumption that skills are normally distributed then this distribution is no longer identified from observations $(Y_i,D_i)$. In fact, as \citet{heckman1990empirical} explain, given a cross-section of data on earnings and occupational choice, an econometrician would be unable to distinguish a model with positively (or negatively) correlated skills from one in which skills are independent. A key distinction between the classical Roy model and the Roy model with matching is that potential earnings $E_i$ are endogenous objects determined in equilibrium. However, as \citet{heckman1985heterogeneity} observe, endogenizing wages isn't enough to obtain identification of the unconditional skill distribution---which in their case is analogous to potential earnings. To solve this problem, they assume skills are normally distributed. The goal in the analysis presented here will be to introduce matching and try to identify the unconditional distribution of potential earnings without any parametric assumptions on the distribution of skills.

\section{Roy model with matching}

The occupational choice problem is the same as that which was described in the previous section. That is, workers choose either role $k$ or role $s$ based on which one provides them a higher wage. However, one key difference is we can no longer assume that wages are given by Equations (\ref{eqn:pi_simple}) and (\ref{eqn:w_simple}). Furthermore, we no longer assume that the distribution of skills $F$ is normal.

Consider a firm which employs exactly one $k$-type worker and exactly one $s$-type worker with a production function $R:\Theta \to \mathbb{R}_+$. Each firm will choose a pair of workers according to
\begin{equation}
	\max_{k^*,s^*} R(k^*,s^*) - \pi(k^*) - w(s^*)
	\label{eqn:firm_prob}
\end{equation}
where free entry implies that profits are zero.

\begin{definition}{Competitive equilibrium}
	In this 
\end{definition}


\bibliographystyle{plainnat}
\bibliography{/home/selliott/Research/bib/matching}
\end{document}

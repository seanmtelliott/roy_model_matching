%%% Document class (e.g. article, letter, beamer, etc...)
\documentclass{article}
%%%

%%% Packages
\usepackage[round]{natbib} %Allows custom options for bibliography management
\usepackage{cite} %Adds bibliograhy functionality like \citet
\usepackage{geometry} %Allows me to set the margins
\usepackage{amsmath} %Adds additional math-related commands
\usepackage{amsfonts} %Adds fonts-within-math-mode functionality
\usepackage{amsthm} %Enables custom theorem environments
\usepackage{bm} %Adds bold math fonts
\usepackage{setspace} %Enables \doublespacing command
\usepackage{parskip} %Removes indentation
\usepackage{tikz} %Used for formatting plots and figures
\usepackage{float} %Used to force the position of tables and figures
\usepackage{graphicx} %For including images
\usepackage[T1]{fontenc} %Allows for encoding of slavic letters
\usepackage{lmodern} %Need to load back the original font for when T1 encoding is active
\usepackage{hyperref}
%%%

%%% Set the margins of the document
\geometry{
	left=1in,
	right=1in,
	bottom=1in,
	top=0.01in,
}
%%%

%%% User-defined commands

\DeclareMathOperator*{\argmax}{argmax} % The argmax operator

\newtheorem{assumption}{Assumption} % Treating assumptions like theorems.
\newtheorem{proposition}{Proposition} % Treating propositions like theorems.

%%%

%%% Title
\title{Identification in linear separation case}
\author{}
\date{}
%%%

\begin{document}
	\vspace{-1.5cm}
	\maketitle
	\vspace{-1.75cm}
	
	Here we consider the case where the revenue function is given by
	\begin{equation*}
		F(k,s) = ak + bs.
	\end{equation*}
	Suppose individual $i$ has skills $(k_i,s_i)$ and log earnings in each occupation $(Y_{k_i},Y_{s_i})$. The individual chooses occupation according to
	\begin{equation*}
	 J_i =	\begin{cases}
		k \text{  if  } Y_{k_i} > Y_{s_i}\\
		s \text{  if  } Y_{k_i} < Y_{s_i}.
		\end{cases}
	\end{equation*}
	
	
	It is possible to show that in this scenario the wage earned by individual $i$ is
	\begin{equation*}
		Y_i =
		\begin{cases}
			\log a + \log k_i \text{ if } J_i = k \\
			\log b + \log s_i \text{ if } J_i = s
		\end{cases}
	\end{equation*}
	where $a$ and $b$ are the skill prices. Also suppose further that $(\ln k_i, \ln s_i) \sim N(\mathbf{\mu},\Sigma)$ where $\mu = [\mu_k,\mu_s]'$ and $\Sigma=\begin{bmatrix}
		\sigma_k^2 & \sigma_{ks} \\
		\sigma_{ks} & \sigma_s^2
	\end{bmatrix}$. 
	
	Thus, the distribution of observed log wages is also normal but the means are linearly augmented by the log of the skill prices. If we denote $\tilde{\mu}_k = \mu_k + \log a$ and $\tilde{\mu}_s = \mu_s + \log b$ and also define 
	\begin{equation*}
		D = \frac{\tilde{\mu}_k - \tilde{\mu}_s}{\sqrt{\sigma_k^2 + \sigma_s^2 - 2\sigma_{ks}}},
	\end{equation*}
	\begin{equation*}
		\tau_h = \frac{\sigma_h^2 - \sigma_{ks}}{\sqrt{\sigma_k^2 + \sigma_s^2 - 2\sigma_{ks}}} \text{ for } h \in \{k,s\},
	\end{equation*}
	and 
	\begin{equation*}
		\lambda(\cdot) = \frac{\varphi(\cdot)}{\Phi(\cdot)}
	\end{equation*}
	where $\varphi$ and $\Phi$ are the pdf and cdf, respectively, of the standard normal distribution. Then from \citet{heckman1990empirical} and following notation from  \citet{french2011identification}, we have the following properties of normal RVs:
	\begin{equation}
		Pr(J_i = k) = \Phi(D)
	\end{equation}
	\begin{equation}
		E(Y_i | J_i = k) = \tilde{\mu}_k + \tau_k \lambda(D)
	\end{equation}
	\begin{equation}
		E(Y_i | J_i = s) = \tilde{\mu}_s + \tau_s \lambda(-D)
	\end{equation}
	\begin{equation}
		Var(Y_i | J_i = k) = \sigma_k^2 + \tau_k^2(-\lambda(D)D - \lambda^2(D))
	\end{equation}
		\begin{equation}
		Var(Y_i | J_i = s) = \sigma_s^2 + \tau_s^2(\lambda(-D)D - \lambda^2(D))
	\end{equation}
	\begin{equation}
		E([Y_i - E(Y_i|J_i = k)]^3|J_i = k) = \tau_k^3 \lambda(D)[2\lambda^2(D) + 3 D \lambda(D) + D^2 -1]
	\end{equation}
		\begin{equation}
		E([Y_i - E(Y_i|J_i = s)]^3|J_i = s) = \tau_s^3 \lambda(-D)[2\lambda^2(-D) - 3 D \lambda(-D) + D^2 -1]
	\end{equation}
	

		Then we can identify $\tilde{\mu}_k$, $\tilde{\mu}_s$, $\sigma_k^2$, $\sigma_s^2$, $\sigma_{ks}$ from Equations (1)--(7). From Equation (1) we get that $D = \Phi^{-1}(Pr(J_i=k))$ which implies that $D$ is known and thus $\lambda(D)$ is known. We can then immediately use this to solve for $\tau_k$ and $\tau_s$ from Equations (6) and (7). Then from here we can get $\tilde{\mu}_k$, $\tilde{\mu}_s$, $\sigma_k^2$, $\sigma_s^2$ from Equations (2)--(5) and then use (1) to get $\sigma_{ks}$. We could also use this to construct an estimator for the parameters.
		
		Then, similarly, we should be able to use the revenue function $F(k,s)$ to get the skill prices $a$ and $b$ in a similar way. I am currently working this out.
		
		

	\bibliographystyle{plainnat}
	\bibliography{/home/selliott/Research/bib/matching}

\end{document}

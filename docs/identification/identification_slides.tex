%%%%%%%%%%%%%%%%%%%%%%%%%%%%%%%%%%%%%%%%%
% Beamer Presentation
% LaTeX Template
% Version 1.0 (10/11/12)
%
% This template has been downloaded from:
% http://www.LaTeXTemplates.com
%
% License:
% CC BY-NC-SA 3.0 (http://creativecommons.org/licenses/by-nc-sa/3.0/)
%
%%%%%%%%%%%%%%%%%%%%%%%%%%%%%%%%%%%%%%%%%

%----------------------------------------------------------------------------------------
%	PACKAGES AND THEMES
%----------------------------------------------------------------------------------------

\documentclass{beamer}
\beamertemplatenavigationsymbolsempty
\let\oldfootnotesize\footnotesize
\renewcommand*{\footnotesize}{\oldfootnotesize\tiny}
\mode<presentation> {
	
	% The Beamer class comes with a number of default slide themes
	% which change the colors and layouts of slides. Below this is a list
	% of all the themes, uncomment each in turn to see what they look like.
	
	%\usetheme{default}
	%\usetheme{AnnArbor}
	%\usetheme{Antibes}
	%\usetheme{Bergen}
	%\usetheme{Berkeley}
	%\usetheme{Berlin}
	\usetheme{Boadilla}
	%\usetheme{CambridgeUS}
	%\usetheme{Copenhagen}
	%\usetheme{Darmstadt}
	%\usetheme{Dresden}
	%\usetheme{Frankfurt}
	%\usetheme{Goettingen}
	%\usetheme{Hannover}
	%\usetheme{Ilmenau}
	%\usetheme{JuanLesPins}
	%\usetheme{Luebeck}
	%\usetheme{Madrid}
	%\usetheme{Malmoe}
	%\usetheme{Marburg}
	%\usetheme{Montpellier}
	%\usetheme{PaloAlto}
	%\usetheme{Pittsburgh}
	%\usetheme{Rochester}
	%\usetheme{Singapore}
	%\usetheme{Szeged}
	%\usetheme{Warsaw}
	
	% As well as themes, the Beamer class has a number of color themes
	% for any slide theme. Uncomment each of these in turn to see how it
	% changes the colors of your current slide theme.
	
	%\usecolortheme{albatross}
	%\usecolortheme{beaver}
	%\usecolortheme{beetle}
	%\usecolortheme{crane}
	%\usecolortheme{dolphin}
	%\usecolortheme{dove}
	%\usecolortheme{fly}
	%\usecolortheme{lily}
	%\usecolortheme{orchid}
	%\usecolortheme{rose}
	%\usecolortheme{seagull}
	%\usecolortheme{seahorse}
	%\usecolortheme{whale}
	%\usecolortheme{wolverine}
	
	%\setbeamertemplate{footline} % To remove the footer line in all slides uncomment this line
	\setbeamertemplate{footline}[page number] % To replace the footer line in all slides with a simple slide count uncomment this line
	
	%\setbeamertemplate{navigation symbols}{} % To remove the navigation symbols from the bottom of all slides uncomment this line
}

%% Packages
\usepackage[round]{natbib} %Allows custom options for bibliography management
\usepackage{cite} %Adds bibliograhy functionality like \citet
\usepackage{amsmath} %Adds additional math-related commands
\usepackage{amsfonts} %Adds fonts-within-math-mode functionality
\usepackage{amsthm} %Enables custom theorem environments
%\usepackage{bm} %Adds bold math fonts
\usepackage{adjustbox} %Need to adjust the size of tikz diagram
\usepackage{tikz} %Used for formatting plots and figures
\usepackage{float} %Used to force the position of tables and figures
\usepackage{graphicx} %For including images
\usepackage[T1]{fontenc} %Allows for encoding of slavic letters
\usepackage{lmodern} %Need to load back the original font for when T1 encoding is active
%%


%%% User-defined commands

\DeclareMathOperator*{\argmax}{argmax} % The argmax operator
\DeclareMathOperator\supp{supp}

\newtheorem{assumption}{Assumption} % Treating assumptions like theorems.
\newtheorem{proposition}{Proposition} % Treating propositions like theorems.
%\newtheorem{problem}{Problem}


%%%

\bibliographystyle{apalike}
% make bibliography entries smaller
\renewcommand\bibfont{\scriptsize}
% If you have more than one page of references, you want to tell beamer
% to put the continuation section label from the second slide onwards
\setbeamertemplate{frametitle continuation}[from second]
% Now get rid of all the colours
\setbeamercolor*{bibliography entry title}{fg=black}
\setbeamercolor*{bibliography entry author}{fg=black}
\setbeamercolor*{bibliography entry location}{fg=black}
\setbeamercolor*{bibliography entry note}{fg=black}
% and kill the abominable icon
\setbeamertemplate{bibliography item}{}


\title[Identification in a Generalized Roy Model with Matching]{Identification in a Generalized Roy Model with Matching}
\date{August 2, 2024}
\author{Sean Elliott}
\institute{University of Toronto}

\begin{document}
	
	
	
%	\begin{frame}
%		\titlepage % Print the title page as the first slide
%	\end{frame}
%	

	\begin{frame}{Model preliminaries}
		
		\begin{itemize}
			\item Each individual's skill level is given by the ordered pair $(k,s)$ where $k\in\mathbb{R}_+$ and $s\in\mathbb{R}_+$ represent the managerial and worker skill, respectively. Denote the distribution of skills as $\mathcal{R}\subseteq \mathbb{R}^2_+$.
			
			\bigskip
			
			\item Firms have a strictly supermodular production function $F:\mathbb{R}^2_+ \to \mathbb{R}$ which depends on the type of workers $k$ and $s$ they employ.
			
			\bigskip
			
			\item Workers choose their role (manager or worker) to maximize their wage and firms choose workers to maximize profits.
			
			\bigskip
			
			\item The goal is to identify the joint distribution of skills $\mathcal{R}$ and the production function $F(k,s)$ from observed wages and matches.

		\end{itemize}
		

	\end{frame}

	\begin{frame}{Worker and firm problem}
		Workers evaluate 
		\begin{equation*}
			\max\{\pi(k),w(s)\}
		\end{equation*}
		where $\pi(k)$ and $w(s)$ are the wage functions for the managerial and worker role, respectively.

	\bigskip
	
	A worker is indifferent between a managerial or worker role if $\pi(k) = w(s)$, or equivalently, if $s=w^{-1}(\pi(k))$.
	
	\bigskip
	
	Firms choose a worker of each type to maximize profit given by
	\begin{equation*}
		\max_{(k,s)\in\supp(\mathcal{R})} F(k,s) - \pi(k) - w(s).
	\end{equation*}
	
	\bigskip
	
	We can now define what a competitive equilibrium looks like in this market.
	
		\end{frame}
		
		
	\begin{frame}{Competitive equilibrium}
		
		\begin{definition}
			Given a production function $F$ and distribution of skills $\mathcal{R}$, a competitive equilibrium is a pair of wage schedules $\pi$ and $w$ such that
			\smallskip
			\begin{itemize}
				\item Individuals maximize their earnings.
				 $$ max\{\pi(k),w(s)\} $$
				\item Firms have free entry and choose a manager and worker to maximize profit --- which is zero in equilibrium.
				$$ \max_{(k,s)\in\supp(\mathcal{R})} F(k,s) - \pi(k) - w(s) $$
				\item The labour market clears: each firm employs exactly one $k$-type worker and one $s$-type worker and all workers are employed. 
			\end{itemize}
		\end{definition}
		
	\end{frame}
	
	\begin{frame}{Recasting the model as an optimal transportation problem}
		We are mapping the distribution of managers to the distribution of workers such that it maximizes the profit of firms and obeys occupational choice constraints for workers.
		
		\bigskip
		
		This is a special case of the Monge-Kantorovich problem from optimal transportation.
		
		\bigskip
		
		As in the Roy model, there is complete separation in the labour market characterized by the separating function 
		\begin{equation*}
			\phi(k) = w^{-1}(\pi(k))
		\end{equation*}
		where the key difference in our case is that we do not restrict $\phi$ to be linear.
		
		\bigskip
		
		 As will be shown later, if we allow interaction between workers in the production function, this generates non-linearity in $\phi$.
	\end{frame}
	
	
	\begin{frame}{Supplies of managers/workers}
		We can use the separating function to define the occupational distributions for managers and workers. Suppose the skill distribution $\mathcal{R}$ has density $R$ then the distributions are given by 
		\begin{equation*}
			H^\phi(k) = \int_0^k\int_0^{\phi(\tilde{k})} R(s,\tilde{k})dsd\tilde{k}
		\end{equation*}
		
		\begin{equation*}
			G^\phi(s) = \int_0^s\int_0^{\phi^{-1}(\tilde{s})} R(\tilde{s},k)dkd\tilde{s}
		\end{equation*}
		Furthermore, from supermodularity of $F$, we have positive assortative matching and there is a matching function $\mu:\mathbb{R}\to\mathbb{R}$ strictly increasing which solves 
		\begin{equation*}
			\pi(k) + w(\mu(k)) = F(k,\mu(k)) \;\; \forall k \in \mathbb{R}_+
		\end{equation*}
		and we have that market clearing is characterized by
		\begin{equation*}
			H^\phi(k) = G^\phi(\mu(k)).
		\end{equation*}
	\end{frame}
	
	\begin{frame}
		\begin{definition}[$F$-transform]
			Given $\Omega_1\subseteq\mathbb{R}_+$ and $\pi:\Omega_1 \to \mathbb{R}$ define $\pi^F$ as the $F$-transform of $\pi$ as
			\begin{equation*}
				\pi^F(s) \equiv \sup_{k\in\Omega_1}\{F(k,s) - \pi(k)\}.
			\end{equation*}
			Similarly, given $\Omega_2\subseteq\mathbb{R}_+$ and $w:\Omega_2 \to \mathbb{R}$ define $w^F$ as the $F$-transform of $\pi$ as
				\begin{equation*}
				w^F(k) \equiv \sup_{s\in\Omega_2}\{F(k,s) - w(s)\}.
			\end{equation*}
		\end{definition}
		
				\begin{definition}[Push-forward measure]
		Given two measurable spaces $(X,\Sigma_1)$ and $(Y,\Sigma_2)$, a measurable function $f:X \to Y$, and a measure $\nu:\Sigma_1 \to \mathbb{R}_+$, the push-forward measure $f\#\nu:\Sigma_2\to\mathbb{R}_+$ is given by
		\begin{equation*}
			f\#\nu(A) = \nu(f^{-1}(A))
		\end{equation*}
		for some $A\in\Sigma_2$.
		\end{definition}
	\end{frame}
	
	\begin{frame}{Social planner's problem}
		\begin{problem}[Generalized Roy Model]
			Given skill distribution $\mathcal{R}$ with density $R:\supp(\mathcal{R}) \to \mathbb{R}$ and a strictly supermodular function $F:\mathbb{R}^2_+ \to \mathbb{R}$, the planner solves the non-linear optimization program
			\begin{equation*}
				\sup_{(\phi,\pi,w,\mu)} \left\{\int F(k,\mu(k))dH^\phi\right\} 
			\end{equation*}
			such that
			\smallskip
			\begin{enumerate}
				\item $\pi = w^F$
				\item $\phi$ generates a 1/2-cut of $\mathcal{R}$ 
				\item $\mu\#H^\phi = G^{\phi}$
				\item $||\pi||_{\infty} \leq ||F||_\infty$
			\end{enumerate}
			\smallskip
			where $||\cdot||_\infty$ denotes the supremum of a function on its domain.
		\end{problem}
	\end{frame}
	
	\begin{frame}{Identification problem}
		\begin{itemize}
			\item If we assume that wages and within-firm matches are generated by the optimization problem on the previous slide, what model primitives can be recovered?
			\bigskip
			\item We assume that we can observe the wage distribution of each occupation, occupational choices of workers, and who is matched with whom.
			\bigskip
			\item The goal is to recover the joint distribution of skills $\mathcal{R}$ and the production function $F(k,s)$.
			\bigskip
			\item As a starting point, we will first show that under the case where $\phi$ is linear  (i.e., no interaction in the production function), the problem reduces to the familiar case of \citet{heckman1990empirical}.
		\end{itemize}
	\end{frame}
	
	\begin{frame}{Case of linear separation}
		Suppose that the production function is given by 
		\begin{equation*}
			F(k,s) = ak + bs
		\end{equation*}
		where $a,b \in \mathbb{R}_+$ are constants. In this scenario, who is matched with whom does not impact production and we have that for any function $\mu$ such that $\mu\#H^\phi = G^\phi$, the objective function becomes
		\begin{equation*}
			\int F(k,\mu(k)) dH^\phi = \int ak \; dH^\phi + \int bs \; dG^\phi
		\end{equation*}
		which implies we can choose any volume-preserving matching function $\mu$.
		
		\bigskip
		
		Furthermore, we can show that this case corresponds exactly to that of \citet{heckman1990empirical} and the classical Roy model.
	\end{frame}
	
	\begin{frame}{Recovering the wage functions}
		It follows from the envelope theorem that for the optimal quadruple $(\phi,\pi,w,\mu)$ from the social planner's problem we may write
		\begin{equation*}
			\pi'(k) = F_1(k,\mu(k)) \text{ and } w'(s) = F_2(\mu^{-1}(s),s)
		\end{equation*}
		where $F_j(\cdot)$ denotes the partial derivative of $F$ w.r.t. the $j$th coordinate.
		
		\bigskip
		\bigskip
		
		By integrating, we can write the wages for the two occupations as
		\begin{equation*}
			\pi(k) = \alpha + \int_0^kF_1(\tilde{k},\mu(\tilde{k})) d\tilde{k}
		\end{equation*}
		and
		\begin{equation*}
			w(s) = \beta + \int_0^sF_2(\mu^{-1}(\tilde{s}),\tilde{s}) d\tilde{s}
		\end{equation*}
		where $\alpha,\beta \in \mathbb{R}$ are integration constants.
	\end{frame}
	
	\begin{frame}{Wage functions with linear $\phi$}
		In the case where $F(k,s) = ak + bs$ we get wage functions
		\begin{equation*}
			\pi(k) = \alpha + ak
		\end{equation*}
		and
		\begin{equation*}
			w(s) = \beta + bs.
		\end{equation*}
		If we assume that $\pi(0) = w(0) = 0$ this implies that $\alpha=\beta=0$ so that $\pi(k)=ak$ and $w(s)=bs$.
		
		\bigskip
		
		Furthermore, if we assume that $(\ln k, \ln s) \sim N(\mathbf{\mu},\Sigma)$ where $\mu = [\mu_k,\mu_s]'$ and $\Sigma=\begin{bmatrix}
			\sigma_k^2 & \sigma_{ks} \\
			\sigma_{ks} & \sigma_s^2
		\end{bmatrix}$ then we have exactly the conditions of \citet{heckman1990empirical}.\\
		
		\bigskip
		
		It is worth recalling exactly the identification results which were established for the log-normal Roy model.
	\end{frame}
	
	\begin{frame}{Identification of the log-normal Roy model}
		\begin{theorem}
			Under the condition of linear separation and log-normal skills, we have the following results from \citet{heckman1990empirical}:
			\begin{itemize}
				\item HH (Thm. 4): If we observe wages for both occupations and occupational choice, both $\mathbf{\mu}$ and $\Sigma$, and their subscripts, are identified.
				\smallskip
				\item  HH (Thm. 5): If we only observe the aggregate earnings distribution, we can identify both $\mathbf{\mu}$ and $\Sigma$ but not their subscripts.
				\smallskip
				\item HH (Thm. 6): If we observe earnings for one occupation, say $k$, it is possible to identify $\mu_k$, $\sigma_k$, $\frac{\mu_k - \mu_s}{\sigma}$ and $\rho$ where $\sigma = \sqrt{\sigma_k + \sigma_s -2 \sigma_{ks}}$ and $\rho = \frac{\sigma_k - \sigma_{ks}}{\sigma\sqrt{\sigma_k}}$. However, we cannot identify $\mu_s$ and $\sigma$.
			\end{itemize}
		\end{theorem}
		For proofs of the first two see \citet{basu1978identifiability} (or \citet{french2011identification} under slightly different conditions). The third proof can be found in \citet{heckman1990empirical}.
	\end{frame}
	
	\begin{frame}{Non-parametric identification}
		\begin{itemize}
			\item Relaxing the normality assumption invalidates the previously established identification results. \citep{heckman1990empirical}
			\smallskip
			\item That is, without additional assumptions, relaxing normality implies that sectoral wage data can be rationalized by a model with skills having arbitrary correlation.
			\smallskip
			\item However, this can be remedied under different approaches:
			\begin{itemize}
				\smallskip
				\item Identification at infinity: \citet{heckman1990empirical} \citet{french2011identification}, \citet{d2013another}
				\smallskip
				\item Adding a non-pecuniary component: \citet{heckman2007econometric} \citet{d2013inference}, \citet{bayer2011nonparametric}, \citet{lee2023nonparametric}.\footnote{It is worth noting here that additional assumptions need to be made to uncover the skill distribution in this case. In fact, \citet{french2011identification} show that even normality is not sufficient to achieve identification.}
				\smallskip
				\item Exclusion restrictions: \citet{heckman1990empirical}, \citet{buera2006non}, \citet{french2011identification}
				\smallskip
				\item Construct bounds on the joint skill distribution:  \citet{mourifie2020sharp}
			\end{itemize}
		\end{itemize}
	\end{frame}
	
	\begin{frame}{Simulations -- Log-normal skill distribution}
		\begin{itemize}
			\item Here I will present some simulations to show how the observed objects (wage and matching functions) change as the skill distribution changes.
			\bigskip
			\item We will relax the restriction that $\phi$ is linear by using the production function $F(k,s) = 0.75k + 0.25s + ks$.
			\bigskip
			\item  Skills are drawn from a bivariate lognormal distribution $(k,s) \sim LN\left(\begin{bmatrix}
				\mu_k \\ 
				\mu_s
			\end{bmatrix}, 
			\begin{bmatrix}
				\sigma^2_{k} & \sigma_{ks} \\
				\sigma_{ks} & \sigma^2_{s}
			\end{bmatrix}\right)$ and then truncated so each pair $(k,s)$ lies in the unit square.
			\bigskip
			\item I will show what happens as we adjust the means of both skills, the correlation between the two skills, only the manager skill mean, and only the manager skill variance.
		\end{itemize}

	\end{frame}
	
	\begin{frame}{Simulations}
		These cases can be summarized as:
		\begin{enumerate}
			\item \textbf{Altering the mean -- both roles}: Fix $\Sigma = \begin{bmatrix}
				1 & 0.5 \\
				0.5 & 1
			\end{bmatrix}$ and consider $\mu_k,\mu_s \in \{-3,-2,-1\}$.
			\item \textbf{Altering the correlation}: Fix $\mu_k = \mu_s = -1$ and consider $\Sigma = \begin{bmatrix}
				1 & \sigma_{ks} \\
				\sigma_{ks} & 1
			\end{bmatrix}$ where $\sigma_{ks}\in\{0.25,0.5,0.75\}$.
			\item \textbf{Altering the mean -- manager skill only}: Fix $\Sigma = \begin{bmatrix}
				1 & 0.5 \\
				0.5 & 1
			\end{bmatrix}$ and $\mu_s = -1$ and consider $\mu_k \in \{-1.5,-1,-0.5\}$.
			\item \textbf{Altering the variance -- manager skill only}: Fix $\mu_k = \mu_s = - 1$ and consider $\Sigma = \begin{bmatrix}
				\sigma^2_k & 0.5 \\
				0.5 & 1
			\end{bmatrix}$ with $\sigma^2_k \in \{1,2,3\}$.
		\end{enumerate}
	\end{frame}
	
%	\begin{frame}{\small Changing $\mu_k$ and $\mu_s$}
%		\includegraphics[width=.3\textwidth]{symmetric/contourmu=-3}\hfill
%		\includegraphics[width=.3\textwidth]{symmetric/contourmu=-2}\hfill
%		\includegraphics[width=.3\textwidth]{symmetric/contourmu=-1}\\
%		\centering
%		\includegraphics[width=0.65\textwidth]{symmetric/mu_varying_obs.png}
%	\end{frame}
%	
%	\begin{frame}{\small Changing $\rho$}
%		\includegraphics[width=.3\textwidth]{symmetric/contourrho=0.25}\hfill
%		\includegraphics[width=.3\textwidth]{symmetric/contourrho=0.5}\hfill
%		\includegraphics[width=.3\textwidth]{symmetric/contourrho=0.75}\\
%		\centering
%		\includegraphics[width=0.65\textwidth]{symmetric/rho_varying_obs.png}
%	\end{frame}
%	
%		\begin{frame}{\small Changing $\mu_k$ and keeping $\mu_s$ fixed}
%		
%		\includegraphics[width=.3\textwidth]{asymmetric/contourmu=-1.5}\hfill
%		\includegraphics[width=.3\textwidth]{asymmetric/contourmu=-1}\hfill
%		\includegraphics[width=.3\textwidth]{asymmetric/contourmu=-0.5}\\
%		\centering
%		\includegraphics[width=0.65\textwidth]{asymmetric/mu_k_varying_obs.png}
%	\end{frame}
%	
%			\begin{frame}{\small Changing $\sigma^2_k$ and keeping $\sigma^2_s$ fixed}
%		
%		\includegraphics[width=.3\textwidth]{asymmetric/contoursigma=1}\hfill
%		\includegraphics[width=.3\textwidth]{asymmetric/contoursigma=2}\hfill
%		\includegraphics[width=.3\textwidth]{asymmetric/contoursigma=3}\\
%		\centering
%		\includegraphics[width=0.65\textwidth]{asymmetric/sigma_k_varying_obs.png}
%	\end{frame}
	
	
	\bibliography{/home/selliott/Research/bib/matching}
	
	
\end{document}

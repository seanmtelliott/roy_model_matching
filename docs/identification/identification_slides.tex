%%%%%%%%%%%%%%%%%%%%%%%%%%%%%%%%%%%%%%%%%
% Beamer Presentation
% LaTeX Template
% Version 1.0 (10/11/12)
%
% This template has been downloaded from:
% http://www.LaTeXTemplates.com
%
% License:
% CC BY-NC-SA 3.0 (http://creativecommons.org/licenses/by-nc-sa/3.0/)
%
%%%%%%%%%%%%%%%%%%%%%%%%%%%%%%%%%%%%%%%%%

%----------------------------------------------------------------------------------------
%	PACKAGES AND THEMES
%----------------------------------------------------------------------------------------

\documentclass{beamer}
\beamertemplatenavigationsymbolsempty
\let\oldfootnotesize\footnotesize
\renewcommand*{\footnotesize}{\oldfootnotesize\tiny}
\mode<presentation> {
	
	% The Beamer class comes with a number of default slide themes
	% which change the colors and layouts of slides. Below this is a list
	% of all the themes, uncomment each in turn to see what they look like.
	
	%\usetheme{default}
	%\usetheme{AnnArbor}
	%\usetheme{Antibes}
	%\usetheme{Bergen}
	%\usetheme{Berkeley}
	%\usetheme{Berlin}
	\usetheme{Boadilla}
	%\usetheme{CambridgeUS}
	%\usetheme{Copenhagen}
	%\usetheme{Darmstadt}
	%\usetheme{Dresden}
	%\usetheme{Frankfurt}
	%\usetheme{Goettingen}
	%\usetheme{Hannover}
	%\usetheme{Ilmenau}
	%\usetheme{JuanLesPins}
	%\usetheme{Luebeck}
	%\usetheme{Madrid}
	%\usetheme{Malmoe}
	%\usetheme{Marburg}
	%\usetheme{Montpellier}
	%\usetheme{PaloAlto}
	%\usetheme{Pittsburgh}
	%\usetheme{Rochester}
	%\usetheme{Singapore}
	%\usetheme{Szeged}
	%\usetheme{Warsaw}
	
	% As well as themes, the Beamer class has a number of color themes
	% for any slide theme. Uncomment each of these in turn to see how it
	% changes the colors of your current slide theme.
	
	%\usecolortheme{albatross}
	%\usecolortheme{beaver}
	%\usecolortheme{beetle}
	%\usecolortheme{crane}
	%\usecolortheme{dolphin}
	%\usecolortheme{dove}
	%\usecolortheme{fly}
	%\usecolortheme{lily}
	%\usecolortheme{orchid}
	%\usecolortheme{rose}
	%\usecolortheme{seagull}
	%\usecolortheme{seahorse}
	%\usecolortheme{whale}
	%\usecolortheme{wolverine}
	
	%\setbeamertemplate{footline} % To remove the footer line in all slides uncomment this line
	\setbeamertemplate{footline}[page number] % To replace the footer line in all slides with a simple slide count uncomment this line
	
	%\setbeamertemplate{navigation symbols}{} % To remove the navigation symbols from the bottom of all slides uncomment this line
}

%% Packages
\usepackage[round]{natbib} %Allows custom options for bibliography management
\usepackage{cite} %Adds bibliograhy functionality like \citet
\usepackage{amsmath} %Adds additional math-related commands
\usepackage{amsfonts} %Adds fonts-within-math-mode functionality
\usepackage{amsthm} %Enables custom theorem environments
%\usepackage{bm} %Adds bold math fonts
\usepackage{adjustbox} %Need to adjust the size of tikz diagram
\usepackage{tikz} %Used for formatting plots and figures
\usepackage{float} %Used to force the position of tables and figures
\usepackage{graphicx} %For including images
\usepackage[T1]{fontenc} %Allows for encoding of slavic letters
\usepackage{lmodern} %Need to load back the original font for when T1 encoding is active
%%


%%% User-defined commands

\DeclareMathOperator*{\argmax}{argmax} % The argmax operator
\DeclareMathOperator\supp{supp}

\newtheorem{assumption}{Assumption} % Treating assumptions like theorems.
\newtheorem{proposition}{Proposition} % Treating propositions like theorems.


%%%

\bibliographystyle{apalike}
% make bibliography entries smaller
\renewcommand\bibfont{\scriptsize}
% If you have more than one page of references, you want to tell beamer
% to put the continuation section label from the second slide onwards
\setbeamertemplate{frametitle continuation}[from second]
% Now get rid of all the colours
\setbeamercolor*{bibliography entry title}{fg=black}
\setbeamercolor*{bibliography entry author}{fg=black}
\setbeamercolor*{bibliography entry location}{fg=black}
\setbeamercolor*{bibliography entry note}{fg=black}
% and kill the abominable icon
\setbeamertemplate{bibliography item}{}


\title[Identification in a Generalized Roy Model with Matching]{Identification in a Generalized Roy Model with Matching}
\date{August 2, 2024}
\author{Sean Elliott}
\institute{University of Toronto}

\begin{document}
	
	
	
%	\begin{frame}
%		\titlepage % Print the title page as the first slide
%	\end{frame}
%	

	\begin{frame}{Model preliminaries}
		
		\begin{itemize}
			\item Each individual's skill level is given by the ordered pair $(k,s)$ where $k\in\mathbb{R}_+$ and $s\in\mathbb{R}_+$ represent the managerial and worker skill, respectively. Denote the distribution of skills as $\mathcal{R}\subseteq \mathbb{R}^2_+$.
			
			\bigskip
			
			\item Firms have a production function $F:\mathbb{R}^2_+ \to \mathbb{R}$ which depends on the type of workers $k$ and $s$ they employ.
			
			\bigskip
			
			\item Workers choose their role (manager or worker) to maximize their wage and firms choose workers to maximize profits.
			
			\bigskip
			
			\item The goal is to identify the joint distribution of skills $\mathcal{R}$ and the production function $F(k,s)$ from observed wages and matches.

		\end{itemize}
		

	\end{frame}

	\begin{frame}{Worker and firm problem}
		Workers evaluate 
		\begin{equation*}
			\max\{\pi(k),w(s)\}
		\end{equation*}
		where $\pi(k)$ and $w(s)$ are the wage functions for the managerial and worker role, respectively.

	\bigskip
	
	A worker is indifferent between a managerial or worker role if $\pi(k) = w(s)$, or equivalently, if $s=w^{-1}(\pi(k))$.
	
	\bigskip
	
	Firms choose a worker of each type to maximize profit given by
	\begin{equation*}
		\max_{(k,s)\in\supp(\mathcal{R})} F(k,s) - \pi(k) - w(s).
	\end{equation*}
	
	\bigskip
	
	We can now define what a competitive equilibrium looks like in this market.
	
		\end{frame}
		
		
	\begin{frame}{Competitive equilibrium}
		
		\begin{definition}
			Given a production function $F$ and distribution of skills $\mathcal{R}$, a competitive equilibrium is a pair of wage schedules $\pi$ and $w$ such that
			\smallskip
			\begin{itemize}
				\item Individuals maximize their earnings.
				 $$ max\{\pi(k),w(s)\} $$
				\item Firms have free entry and choose a manager and worker to maximize profit --- which is zero in equilibrium.
				$$ \max_{(k,s)\in\supp(\mathcal{R})} F(k,s) - \pi(k) - w(s) $$
				\item The labour market clears: each firm employs exactly one $k$-type worker and one $s$-type worker and all workers are employed. 
			\end{itemize}
		\end{definition}
		
	\end{frame}
	
	\begin{frame}{Recasting the model as an optimal transportation problem}
		We are mapping the distribution of managers to the distribution of workers such that it maximizes the profit of firms and obeys occupational choice constraints for workers.
		
		\bigskip
		
		This is a special case of the Monge-Kantorovich problem from optimal transportation.
		
		\bigskip
		
		As in the Roy model, there is complete separation in the labour market defined by the separating function 
		\begin{equation*}
			\phi(k) = w^{-1}(\pi(k))
		\end{equation*}
		where the key difference in our case is that we do not restrict $\phi$ to be linear.
		
		\bigskip
		
		 As will be shown later, if we allow interaction between workers in the production function, this generates non-linearity in $\phi$.
	\end{frame}
	
	
	\begin{frame}{Supplies of managers/workers}
		We can use the separating function to define the occupational distributions for managers and workers. Suppose the skill distribution $\mathcal{R}$ has density $R$ then the distributions are given by 
		\begin{equation*}
			H^\phi(k) = \int_0^k\int_0^{\phi(\tilde{k})} R(s,\tilde{k})dsd\tilde{k}
		\end{equation*}
		
		\begin{equation*}
			G^\phi(s) = \int_0^s\int_0^{\phi^{-1}(\tilde{s})} R(\tilde{s},k)dkd\tilde{s}
		\end{equation*}
	\end{frame}
	\bibliography{/home/selliott/Research/bib/matching}
	
	
\end{document}

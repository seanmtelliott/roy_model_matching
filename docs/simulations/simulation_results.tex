%%% Document class (e.g. article, letter, beamer, etc...)
\documentclass{article}
%%%

%%% Packages
\usepackage[round]{natbib} %Allows custom options for bibliography management
\usepackage{cite} %Adds bibliograhy functionality like \citet
\usepackage{geometry} %Allows me to set the margins
\usepackage{amsmath} %Adds additional math-related commands
\usepackage{amsfonts} %Adds fonts-within-math-mode functionality
\usepackage{amsthm} %Enables custom theorem environments
\usepackage{bm} %Adds bold math fonts
\usepackage{setspace} %Enables \doublespacing command
\usepackage{parskip} %Removes indentation
\usepackage{tikz} %Used for formatting plots and figures
\usepackage{float} %Used to force the position of tables and figures
\usepackage{graphicx} %For including images
\usepackage[T1]{fontenc} %Allows for encoding of slavic letters
\usepackage{lmodern} %Need to load back the original font for when T1 encoding is active
\usepackage{hyperref} %Add url functionality
%%%

%%% Set the margins of the document
\geometry{
	left=1in,
	right=1in,
	bottom=1in,
	top=1in,
}
%%%

%%% User-defined commands

\DeclareMathOperator*{\argmax}{argmax} % The argmax operator

\newtheorem{assumption}{Assumption} % Treating assumptions like theorems.
\newtheorem{proposition}{Proposition} % Treating propositions like theorems.

%%%

%%% Title
\title{Simulation Results\footnote{Code: \url{https://github.com/seanmtelliott/roy_model_matching}}}
\author{}
\date{}
%%%

\begin{document}
	\maketitle
	%\doublespacing
	%\interfootnotelinepenalty=10000
	
	\section{Simulation overview}
	
	In these simulations the revenue function is given by $F(k,s) = 3k^2 + s^2 + cks$ where $c$ is allowed to vary such that $c\in\{0.01,1.5,3\}$. Two different types of skill distributions are used. The first is a simple grid on the unit square where each worker's skills can be represented by the ordered pair $(k,s) = (n/N,m/N)$ for $n,m = 0,1,2,\cdots, N$ and it is such that each skill set $(k,s)$ is unique. This square grid simulation is the one already in the paper. In the second case, we set $(k,s) \sim \text{Lognormal}[\mathbf{\alpha},\Sigma]$ restricted to the unit square with $\mathbf{\alpha} = (0.5, 0.5)^\prime$ fixed and $\Sigma = \begin{bmatrix}
		1 & \rho \\
		\rho & 1 
	\end{bmatrix}$ where the correlation $\rho$ is positive and varies across the simulations.  \\

	It is also worth noting that I can produce ``nicer'' looking plots (i.e., ggplot or tikz) once we've decided on which ones we want to include in the paper. Additionally, if we want to do something else entirely, the code is sufficiently general such that it allows for $(k,s)$ to be distributed in any way that we choose. 

	\section{Results}
	\subsection{Square grid skills}
	
	\begin{figure}[H]
		\centering
		\includegraphics[scale=0.75]{grid_plot.png}
		\caption{Simulation for $c\in\{0.01,1.5,3\}$ -- Square grid skills}
	\end{figure}

This is a replication of the results already in the paper with the discrepancy being that I am showing the wage inequality as a difference instead of a ratio. That is, the wage inequality is given by $\pi(k) - w(\mu(k))$. Aside from that, the above results are functionally identical to Jeffrey's. Also, I could show inequality as the wage ratio instead, if that is preferred.

\subsection{Lognormal skills}

Here I will show several different simulation results. In the first three simulations I vary $c$ for fixed values of $\rho$. In the last simulation I fix $c=1.5$ and allow $\rho$ to vary.
		\begin{figure}[H]
		\centering
		\includegraphics[scale=0.75]{ln_plot_rho0_10.png}
		\caption{Simulation for $\rho = 0.1$ and $c\in\{0.01,1.5,3\}$ -- Lognormal skills}
	\end{figure}
	\begin{figure}[H]
	\centering
		\includegraphics[scale=0.75]{ln_plot_rho0_5.png}
\caption{Simulation for $\rho = 0.5$ and $c\in\{0.01,1.5,3\}$ -- Lognormal skills}
\end{figure}
	\begin{figure}[H]
	\centering
		\includegraphics[scale=0.75]{ln_plot_rho0_70.png}
\caption{Simulation for $\rho = 0.7$ and $c\in\{0.01,1.5,3\}$ -- Lognormal skills}
\end{figure}

In general, the value of $c$ has less of an impact on the outcome when $(k,s)$ is drawn from the lognormal distribution. We can also ask what happens if we fix $c$ and vary $\rho$.

	\begin{figure}[H]
	\centering
	\includegraphics[scale=0.75]{ln_plot_rho_varying.png}
	\caption{Simulation for $\rho \in\{0.01,0.2,0.5,0.8,0.99\}$ and $c=1.5$ -- Lognormal skills}
\end{figure}

Here we can observe strict convexity in the matching function for the cases where $\rho = 0.8$ and $\rho = 0.99$ which then corresponds to concavity in the wage inequality function. 
	
\end{document}
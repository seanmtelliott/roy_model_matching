%%% Document class (e.g. article, letter, beamer, etc...)
\documentclass{article}
%%%

%%% Packages
\usepackage[round]{natbib} %Allows custom options for bibliography management
\usepackage{cite} %Adds bibliograhy functionality like \citet
\usepackage{geometry} %Allows me to set the margins
\usepackage{amsmath} %Adds additional math-related commands
\usepackage{amsfonts} %Adds fonts-within-math-mode functionality
\usepackage{amsthm} %Enables custom theorem environments
\usepackage{bm} %Adds bold math fonts
\usepackage{setspace} %Enables \doublespacing command
\usepackage{parskip} %Removes indentation
\usepackage{tikz} %Used for formatting plots and figures
\usepackage{float} %Used to force the position of tables and figures
\usepackage{graphicx} %For including images
\usepackage[T1]{fontenc} %Allows for encoding of slavic letters
\usepackage{lmodern} %Need to load back the original font for when T1 encoding is active
\usepackage{hyperref} %Add url functionality
%%%

%%% Set the margins of the document
\geometry{
	left=1in,
	right=1in,
	bottom=1in,
	top=1in,
}
%%%

%%% User-defined commands

\DeclareMathOperator*{\argmax}{argmax} % The argmax operator

\newtheorem{assumption}{Assumption} % Treating assumptions like theorems.
\newtheorem{proposition}{Proposition} % Treating propositions like theorems.

%%%

%%% Title
\title{Separating function simulations}
\author{}
\date{}
%%%

\begin{document}
	%\maketitle
	%\doublespacing
	%\interfootnotelinepenalty=10000
	
\section{Square grid skills}

Using the revenue function from the paper $F(k,s) = ak^2 + bs^2 + cks$ where $a=3,b=1$ and $c \in \{0.01,0.5,2\}$.

	\begin{figure}[H]
	\centering
	\includegraphics[scale=0.75]{grid_plot_paper.png}
	\caption{Simulation for $F(k,s) = 3k^2 + s^2 + cks$ with $c\in\{0.01,0.5,2\}$ -- Square grid skills}
\end{figure}

Notice something strange is occurring here: $\phi$ becomes more linear as $c$ increases. This is the opposite of our expectation.\\

If we instead use the revenue function $F(k,s) = 3k + s + cks$ then we get what we should expect.

	\begin{figure}[H]
	\centering
	\includegraphics[scale=0.75]{grid_plot_linear_ks.png}
	\caption{Simulation for $F(k,s) = 3k + s + cks$ with $c\in\{0.01,0.5,2\}$ -- Square grid skills}
\end{figure}

Now we can see that $\phi$ is indeed linear, or very close to being linear, when $c=0.01$ and there is a bit of non-linearity when $c=2$.\\

We can also consider a case where $c=5$ is the max and the difference is a little more pronounced.

	\begin{figure}[H]
	\centering
	\includegraphics[scale=0.75]{grid_plot_linear_ks_extreme.png}
	\caption{Simulation for $F(k,s) = 3k + s + cks$ with $c\in\{0.01,0.5,5\}$ -- Square grid skills}
\end{figure}

\section{Lognormal skills}
What if we tune the simulation to be similar to that of Heckman and Honore? That is we have skills distributed as $(k,s) \sim \text{Lognormal}[\mathbf{\alpha},\Sigma]$ restricted to the unit square with $\Sigma = \begin{bmatrix}
	\sigma_{11} & \sigma_{12} \\
	\sigma_{21} & \sigma_{22}
\end{bmatrix}$ where we impose that $\sigma_{11} > \sigma_{12}$. We can fix $\sigma_{11}$ and change $\rho$ to get different variations of this.

	\begin{figure}[H]
	\centering
	\includegraphics[scale=0.75]{ln_plot_rho0_05.png}
	\caption{Simulation for $F(k,s) = 3k + s + cks$ with $c\in\{0.01,0.5,2\}$ -- Lognormal ($\rho=0.05$)}
\end{figure}

We get a similar result to the grid case, with a bit more non-linearity visible in $\phi$ when $c$ large. Then increasing $\rho$ we get:

	\begin{figure}[H]
	\centering
	\includegraphics[scale=0.75]{ln_plot_rho0_5.png}
	\caption{Simulation for $F(k,s) = 3k + s + cks$ with $c\in\{0.01,0.5,2\}$ -- Lognormal ($\rho=0.5$)}
\end{figure}

	\begin{figure}[H]
	\centering
	\includegraphics[scale=0.75]{ln_plot_rho0_95.png}
	\caption{Simulation for $F(k,s) = 3k + s + cks$ with $c\in\{0.01,0.5,2\}$ -- Lognormal ($\rho=0.95$)}
\end{figure}

Again, we get similar results with the most non-linearity visible in $\phi$ in the case where $\rho = 0.95$ and $c=2$. 

\end{document}